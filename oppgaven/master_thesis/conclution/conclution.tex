\section{Conclusion}

% talk about how this study answers you research questions

Soil temperature significantly impacts agriculture, influencing pest prevention, conservation, yield prediction, and more. Despite its importance, widespread measurement remains challenging due to cost limitations and technical issues. Interpolating missing data using methods like global mean approximation is common but has drawbacks including requiring a previus messurements of the soil temperature. Incorporating exogenous features can improve soil temperature estimation. Advancements in prediction and measurement are crucial for sustainable agriculture and accurate climate models.

Methods used in this study are 

The prediction of soil temperatures is an important subject of study for the civil population, and the scientific community. For that reason a good model that can be generalised and applied globally is an desired thing.

The resutls show that good results can be achived with few parameters, however furter studies need to be done to see the effekt of adding more parameters. As for modeling; Adding time to a regression model does improve the model predictive power over a time independent model. 

There is a clear advantage to further investigation into deep learning models as the models shows good results, as is show in other studies\cite{feng_estimation_2019,li_attention-aware_2022,li_modeling_2020}. 

\subsection{Limitations}

This study faced a multitude of technical difficulties including,
\begin{itemize}
	\item Getting the models to run
	\item Finding proper parameters
	\item Insufficient computing power
	\item Getting TensorFlow to work
\end{itemize}

\section{Acknowledgements}
\subsection[Use of AI]{Use of Artificial Intelligence in this paper}

In this paper there has been used Artificial Intelligence (AI), specifically Bing Chat / Copilot hosted by Microsoft Cooperation with special agreement with \acrfull{ac:nmbu}, for the following purposes:

\begin{enumerate}
	\item Formalising sentences and rephrasing sentences.
	\item Spellchecking
	\item Code generation of basic consepts and structures (tree traversal, template for generic classes) 
\end{enumerate}

It is important to emphasize that our engagement with AI have been actively curated and verified with known information. All code underwent rigorous manual inspection within a dedicated testing environment. Furthermore, no confidential or sensitive information was shared with the AI; our interactions focused solely on broad topics and general inquiries. To validate the accuracy of AI-generated responses, we cross-referenced them with established research papers and textbooks.
\label{pg:LastBread}