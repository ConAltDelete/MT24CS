\section{Conclusion}

% talk about how this study answers you research questions

Soil temperature significantly impacts agriculture, influencing pest prevention, conservation, yield prediction, and more. Despite its importance, widespread measurement remains challenging due to cost limitations and technical issues. Interpolating missing data using methods like global mean approximation is common but has drawbacks including requiring a previus measurements of the soil temperature. Incorporating exogenous features can improve soil temperature estimation. Advancements in prediction and measurement are crucial for sustainable agriculture and accurate climate models.

Method used in this study are
\begin{enumerate}
	\item Fetch avalible data to use as training and testing set
	\item Compile the data and treat it to be used in the modelig
	\item Train all the models on the same traning data (2014 to 2020)
	\item calculate and plot relavent statitics (2021 to 2022)
	\item compile the results
\end{enumerate} 

The results of this study show that promising results can be achieved with few parameters, however further studies need to be done to see the effekt of adding more parameters or making the models more complex by adding more structure\footnote{Sturcture as in more layers, augmentations of input and features from feature extraction.}. As for regression modeling; Adding time to a regression model does improve the model predictive power over a time independent model. This make a simple model to predict soil temperatures in areas with no soil temperature measurement. 

There is a clear advantage to data-driven modeling to further the investigation into deep learning models as the models shows comparable results to analytical models, as is show in other studies\cite{feng_estimation_2019,li_attention-aware_2022,li_modeling_2020}. However to improve the model performance even more more features is required to make the model more general and adaptible to other local enviroments outside of the nortic climate. 

\subsection{Limitations}

This study faced a multitude of technical difficulties including,
\begin{itemize}
	\item Trying to get the models to run properly
	\item Finding proper parameters to limit memory usage and time
	\item Insufficient computing power for some parameters
	\item Getting TensorFlow to work on Windows 10/11\footnote{This has later in the study that was futile as the project had moved from being Windows compatible to being exclusively runnable on Linux unless choosing an outdated version of Tensorflow.}
\end{itemize}
This study would be more comprehensive if these unforseen difficulties did not occure, and it is recommended that it would be done a new study going more in depth into other type of deep learning models suitable for time-series data.


\section{Acknowledgements}
The progenitor of this study presently partakes in an interval of non-occupational repose, traditionally associated with restorative diversion from vocational pursuits as of immediate effect of thesis submission.
\label{pg:LastBread}