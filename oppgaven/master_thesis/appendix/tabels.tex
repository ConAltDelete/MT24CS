\section{Tables}

%\input{"../../results/tables/"}

\subsection{Model performance}\label{apx:tables:model:summary}

\begin{table}[H]
	\input{"../../results/tables/table_Plauborg_stat_20"}
	\caption[Plauborg-hour-model results for 20cm depth]{Results from hourly version of the Plauborg model for 20cm depth. The station names can be found in table \ref{tab:station:names}.}
	\label{tab:Plauborg:hour:20}
\end{table}
\begin{table}[H]
	\input{"../../results/tables/table_Plauborg_stat_10"}
	\caption[Plauborg-hour-model results for 10cm depth]{Results from hourly version of the Plauborg model for 10cm depth. The station names can be found in table \ref{tab:station:names}.}
	\label{tab:Plauborg:hour:10}
\end{table}
\begin{table}[H]
	\input{"../../results/tables/table_Plauborg_day_stat_20"}
	\caption[Plauborg-day-model results for 20cm depth]{Results from daily version of the Plauborg model for 20cm depth. The station names can be found in table \ref{tab:station:names}.}
	\label{tab:Plauborg:day:20}
\end{table}
\begin{table}[H]
	\input{"../../results/tables/table_Plauborg_day_stat_10"}
	\caption[Plauborg-day-model results for 10cm depth]{Results from daily version of the Plauborg model for 10cm depth. The station names can be found in table \ref{tab:station:names}.}
	\label{tab:Plauborg:day:10}
\end{table}
\begin{table}[H]
	\input{"../../results/tables/table_lin_stat_20"}
	\caption[Linear Regression results for 20cm depth]{Results from the Linear Regression model for 20cm depth. The station names can be found in table \ref{tab:station:names}.}
	\label{tab:linreg:20}
\end{table}
\begin{table}[H]
	\input{"../../results/tables/table_lin_stat_10"}
	\caption[Linear Regression results for 10cm depth]{Results from the Linear Regression model for 10cm depth. The station names can be found in table \ref{tab:station:names}.}
	\label{tab:linreg:10}
\end{table}

\begin{table}[H]
	\input{"tables/table_KerasGRU_stat_20"}
	\caption[GRU results for 20cm depth]{Results from the GRU model for 20cm depth for each region, and station. The "Global" category covers data from all monitoring stations, "Regional" uses stations within a specific area, and "Local" details the results from individual stations. The station names can be found in table \ref{tab:station:names}.}
	\label{tab:gru:20}
\end{table}
\begin{table}[H]
	\input{"tables/table_KerasGRU_stat_10"}
	\caption[GRU results for 10cm depth]{Results from the GRU model for 10cm depth for each region, and station. The "Global" category covers data from all monitoring stations, "Regional" uses stations within a specific area, and "Local" details the results from individual stations. The station names can be found in table \ref{tab:station:names}.}
	\label{tab:GRU:10}
\end{table}
\begin{table}[H]
	\input{"tables/table_KerasBiGRU_stat_20"}
	\caption[BiGRU results for 20cm depth]{Results from the BiGRU model for 20cm depth for each region, and station. The "Global" category covers data from all monitoring stations, "Regional" uses stations within a specific area, and "Local" details the results from individual stations. The station names can be found in table \ref{tab:station:names}.}
	\label{tab:Bigru:20}
\end{table}
\begin{table}[H]
	\input{"tables/table_KerasBiGRU_stat_10"}
	\caption[BiGRU results for 10cm depth]{Results from the BiGRU model for 10cm depth for each region, and station. The "Global" category covers data from all monitoring stations, "Regional" uses stations within a specific area, and "Local" details the results from individual stations. The station names can be found in table \ref{tab:station:names}.}
	\label{tab:BiGRU:10}
\end{table}

\begin{table}[H]
	\input{"tables/table_l1KerasBiLSTM_stat_20"}
	\caption[BiLSTM results for 20cm depth]{Results from the BiLSTM model for 20cm depth for each region, and station. The "Global" category covers data from all monitoring stations, "Regional" uses stations within a specific area, and "Local" details the results from individual stations. The station names can be found in table \ref{tab:station:names}.}
	\label{tab:bilstm:20}
\end{table}
\begin{table}[H]
	\input{"tables/table_l1KerasBiLSTM_stat_10"}
	\caption[BiLSTM results for 10cm depth]{Results from the BiLSTM model for 10cm depth for each region, and station. The "Global" category covers data from all monitoring stations, "Regional" uses stations within a specific area, and "Local" details the results from individual stations. The station names can be found in table \ref{tab:station:names}.}
	\label{tab:bilstm:10}
\end{table}
\begin{table}[H]
	\input{"tables/table_l2KerasBiLSTM_stat_20"}
	\caption[LSTM results for 20cm depth]{Results from the LSTM model for 20cm depth for each region, and station. The "Global" category covers data from all monitoring stations, "Regional" uses stations within a specific area, and "Local" details the results from individual stations. The station names can be found in table \ref{tab:station:names}.}
	\label{tab:lstm:20}
\end{table}
\begin{table}[H]
	\input{"tables/table_l2KerasBiLSTM_stat_10"}
	\caption[LSTM results for 10cm depth]{Results from the LSTM model for 10cm depth for each region, and station. The "Global" category covers data from all monitoring stations, "Regional" uses stations within a specific area, and "Local" details the results from individual stations. The station id can be found in table \ref{tab:station:names}}
	\label{tab:lstm:10}
\end{table}

\subsection{Data summary}\label{apx:tables:station:summary}

\begin{table}
	\begin{tabular}{rllll}
\hline
      & 38   & 34   & 27   & 42   \\
\hline
 2014 & \begin{tabular}{ll}
\hline
 $\mu$:11.71  & max:31.8 \\
 $\sigma$:6.4 & min:-4.0 \\
\hline
\end{tabular}      & \begin{tabular}{ll}
\hline
 $\mu$:10.54   & max:31.4  \\
 $\sigma$:6.89 & min:-14.5 \\
\hline
\end{tabular}      & \begin{tabular}{ll}
\hline
 $\mu$:10.99   & max:33.0 \\
 $\sigma$:6.56 & min:-4.2 \\
\hline
\end{tabular}      & \begin{tabular}{ll}
\hline
 $\mu$:12.0    & max:32.2 \\
 $\sigma$:6.46 & min:-3.1 \\
\hline
\end{tabular}      \\
 2015 & \begin{tabular}{ll}
\hline
 $\mu$:10.38   & max:25.0 \\
 $\sigma$:5.69 & min:-5.7 \\
\hline
\end{tabular}      & \begin{tabular}{ll}
\hline
 $\mu$:9.21    & max:27.4 \\
 $\sigma$:5.42 & min:-5.9 \\
\hline
\end{tabular}      & \begin{tabular}{ll}
\hline
 $\mu$:9.42    & max:27.1 \\
 $\sigma$:5.91 & min:-6.8 \\
\hline
\end{tabular}      & \begin{tabular}{ll}
\hline
 $\mu$:10.42   & max:25.4 \\
 $\sigma$:5.87 & min:-4.8 \\
\hline
\end{tabular}      \\
 2016 & \begin{tabular}{ll}
\hline
 $\mu$:11.34   & max:27.4 \\
 $\sigma$:5.92 & min:-5.2 \\
\hline
\end{tabular}      & \begin{tabular}{ll}
\hline
 $\mu$:9.34   & max:29.4 \\
 $\sigma$:6.1 & min:-8.4 \\
\hline
\end{tabular}      & \begin{tabular}{ll}
\hline
 $\mu$:10.1    & max:27.8 \\
 $\sigma$:6.51 & min:-6.6 \\
\hline
\end{tabular}      & \begin{tabular}{ll}
\hline
 $\mu$:11.04   & max:29.4 \\
 $\sigma$:6.48 & min:-4.2 \\
\hline
\end{tabular}      \\
 2017 & \begin{tabular}{ll}
\hline
 $\mu$:10.35  & max:25.2 \\
 $\sigma$:6.1 & min:-9.2 \\
\hline
\end{tabular}      & \begin{tabular}{ll}
\hline
 $\mu$:8.88    & max:25.6 \\
 $\sigma$:5.99 & min:-9.5 \\
\hline
\end{tabular}      & \begin{tabular}{ll}
\hline
 $\mu$:9.32    & max:29.7  \\
 $\sigma$:6.34 & min:-10.8 \\
\hline
\end{tabular}      & \begin{tabular}{ll}
\hline
 $\mu$:10.56   & max:26.8 \\
 $\sigma$:6.21 & min:-8.5 \\
\hline
\end{tabular}      \\
 2018 & \begin{tabular}{ll}
\hline
 $\mu$:11.03   & max:32.3  \\
 $\sigma$:8.81 & min:-19.7 \\
\hline
\end{tabular}      & \begin{tabular}{ll}
\hline
 $\mu$:9.29    & max:31.6  \\
 $\sigma$:7.73 & min:-19.5 \\
\hline
\end{tabular}      & \begin{tabular}{ll}
\hline
 $\mu$:9.69    & max:31.1  \\
 $\sigma$:9.47 & min:-23.2 \\
\hline
\end{tabular}      & \begin{tabular}{ll}
\hline
 $\mu$:11.4    & max:33.0  \\
 $\sigma$:8.93 & min:-14.5 \\
\hline
\end{tabular}      \\
 2019 & \begin{tabular}{ll}
\hline
 $\mu$:10.94   & max:31.6  \\
 $\sigma$:6.83 & min:-12.0 \\
\hline
\end{tabular}      & \begin{tabular}{ll}
\hline
 $\mu$:8.94    & max:32.7  \\
 $\sigma$:7.02 & min:-14.3 \\
\hline
\end{tabular}      & \begin{tabular}{ll}
\hline
 $\mu$:9.56    & max:31.1  \\
 $\sigma$:7.08 & min:-15.9 \\
\hline
\end{tabular}      & \begin{tabular}{ll}
\hline
 $\mu$:10.47   & max:30.9 \\
 $\sigma$:6.82 & min:-9.8 \\
\hline
\end{tabular}      \\
 2020 & \begin{tabular}{ll}
\hline
 $\mu$:11.28   & max:29.7 \\
 $\sigma$:7.45 & min:-6.8 \\
\hline
\end{tabular}      & \begin{tabular}{ll}
\hline
 $\mu$:9.05   & max:31.3 \\
 $\sigma$:7.0 & min:-7.5 \\
\hline
\end{tabular}      & \begin{tabular}{ll}
\hline
 $\mu$:10.33   & max:30.3  \\
 $\sigma$:6.64 & min:-13.6 \\
\hline
\end{tabular}      & \begin{tabular}{ll}
\hline
 $\mu$:11.03   & max:29.4 \\
 $\sigma$:6.67 & min:-7.5 \\
\hline
\end{tabular}      \\
 2021 & \begin{tabular}{ll}
\hline
 $\mu$:11.72   & max:30.3 \\
 $\sigma$:6.79 & min:-6.1 \\
\hline
\end{tabular}      & \begin{tabular}{ll}
\hline
 $\mu$:9.72    & max:31.7 \\
 $\sigma$:6.48 & min:-6.5 \\
\hline
\end{tabular}      & \begin{tabular}{ll}
\hline
 $\mu$:10.46   & max:29.3 \\
 $\sigma$:6.83 & min:-6.1 \\
\hline
\end{tabular}      & \begin{tabular}{ll}
\hline
 $\mu$:11.35   & max:29.2 \\
 $\sigma$:6.75 & min:-5.8 \\
\hline
\end{tabular}      \\
 2022 & \begin{tabular}{ll}
\hline
 $\mu$:11.33   & max:29.1 \\
 $\sigma$:6.79 & min:-6.9 \\
\hline
\end{tabular}      & \begin{tabular}{ll}
\hline
 $\mu$:8.99    & max:29.2 \\
 $\sigma$:5.87 & min:-7.9 \\
\hline
\end{tabular}      & \begin{tabular}{ll}
\hline
 $\mu$:9.87    & max:27.8  \\
 $\sigma$:6.89 & min:-11.6 \\
\hline
\end{tabular}      & \begin{tabular}{ll}
\hline
 $\mu$:11.09   & max:28.6 \\
 $\sigma$:6.68 & min:-6.3 \\
\hline
\end{tabular}      \\
\hline
\end{tabular}
	\caption[Table of station statistics for air part 1]{The table shows the statistics of each station for each feature, except for Time as it is a strictly monotonic increasing sequence, for each year. The station names can be found in table \ref{tab:station:names}. $\mu$ Denotes the mean temperature, $\sigma$ denotes the standard deviation, "min" is the minimum temperature, and "max" is the maximum temperature. All values in the table have the unit degree Celsius.}
\end{table}
\begin{table}
		\begin{tabular}{rllll}
\hline
      & 38   & 34   & 27   & 42   \\
\hline
 2014 & \begin{tabular}{ll}
\hline
 $\mu$:12.79   & max:22.8 \\
 $\sigma$:5.56 & min:1.5  \\
\hline
\end{tabular}      & \begin{tabular}{ll}
\hline
 $\mu$:9.23    & max:18.4 \\
 $\sigma$:5.27 & min:-0.1 \\
\hline
\end{tabular}      & \begin{tabular}{ll}
\hline
 $\mu$:12.24   & max:23.9 \\
 $\sigma$:5.92 & min:0.9  \\
\hline
\end{tabular}      & \begin{tabular}{ll}
\hline
 $\mu$:12.3    & max:21.8 \\
 $\sigma$:4.89 & min:2.7  \\
\hline
\end{tabular}      \\
 2015 & \begin{tabular}{ll}
\hline
 $\mu$:10.99   & max:18.7 \\
 $\sigma$:5.32 & min:0.4  \\
\hline
\end{tabular}      & \begin{tabular}{ll}
\hline
 $\mu$:8.78    & max:15.4 \\
 $\sigma$:4.09 & min:0.6  \\
\hline
\end{tabular}      & \begin{tabular}{ll}
\hline
 $\mu$:10.5    & max:21.5 \\
 $\sigma$:5.58 & min:-0.2 \\
\hline
\end{tabular}      & \begin{tabular}{ll}
\hline
 $\mu$:10.96   & max:20.3 \\
 $\sigma$:5.29 & min:0.1  \\
\hline
\end{tabular}      \\
 2016 & \begin{tabular}{ll}
\hline
 $\mu$:11.67   & max:19.9 \\
 $\sigma$:6.07 & min:0.1  \\
\hline
\end{tabular}      & \begin{tabular}{ll}
\hline
 $\mu$:8.69    & max:17.0 \\
 $\sigma$:4.85 & min:-0.1 \\
\hline
\end{tabular}      & \begin{tabular}{ll}
\hline
 $\mu$:10.58   & max:20.3 \\
 $\sigma$:6.46 & min:-2.7 \\
\hline
\end{tabular}      & \begin{tabular}{ll}
\hline
 $\mu$:11.29   & max:20.7 \\
 $\sigma$:6.04 & min:-0.1 \\
\hline
\end{tabular}      \\
 2017 & \begin{tabular}{ll}
\hline
 $\mu$:10.55   & max:17.6 \\
 $\sigma$:5.47 & min:0.1  \\
\hline
\end{tabular}      & \begin{tabular}{ll}
\hline
 $\mu$:8.71    & max:17.1 \\
 $\sigma$:4.78 & min:0.1  \\
\hline
\end{tabular}      & \begin{tabular}{ll}
\hline
 $\mu$:9.84    & max:18.8 \\
 $\sigma$:6.07 & min:-1.6 \\
\hline
\end{tabular}      & \begin{tabular}{ll}
\hline
 $\mu$:10.46   & max:19.0 \\
 $\sigma$:5.74 & min:-0.3 \\
\hline
\end{tabular}      \\
 2018 & \begin{tabular}{ll}
\hline
 $\mu$:11.12   & max:21.2 \\
 $\sigma$:6.32 & min:0.5  \\
\hline
\end{tabular}      & \begin{tabular}{ll}
\hline
 $\mu$:8.44    & max:17.6 \\
 $\sigma$:5.49 & min:-1.3 \\
\hline
\end{tabular}      & \begin{tabular}{ll}
\hline
 $\mu$:10.03   & max:22.0 \\
 $\sigma$:6.86 & min:-0.2 \\
\hline
\end{tabular}      & \begin{tabular}{ll}
\hline
 $\mu$:11.05   & max:22.0 \\
 $\sigma$:6.59 & min:0.0  \\
\hline
\end{tabular}      \\
 2019 & \begin{tabular}{ll}
\hline
 $\mu$:10.96   & max:21.0 \\
 $\sigma$:5.67 & min:0.4  \\
\hline
\end{tabular}      & \begin{tabular}{ll}
\hline
 $\mu$:9.08    & max:19.3 \\
 $\sigma$:5.11 & min:-0.2 \\
\hline
\end{tabular}      & \begin{tabular}{ll}
\hline
 $\mu$:10.5    & max:23.1 \\
 $\sigma$:6.15 & min:0.2  \\
\hline
\end{tabular}      & \begin{tabular}{ll}
\hline
 $\mu$:10.67   & max:21.4 \\
 $\sigma$:6.06 & min:0.1  \\
\hline
\end{tabular}      \\
 2020 & \begin{tabular}{ll}
\hline
 $\mu$:9.97    & max:19.5 \\
 $\sigma$:5.87 & min:0.8  \\
\hline
\end{tabular}      & \begin{tabular}{ll}
\hline
 $\mu$:8.53    & max:17.0 \\
 $\sigma$:4.84 & min:0.3  \\
\hline
\end{tabular}      & \begin{tabular}{ll}
\hline
 $\mu$:10.57  & max:21.9 \\
 $\sigma$:6.1 & min:-0.9 \\
\hline
\end{tabular}      & \begin{tabular}{ll}
\hline
 $\mu$:13.18   & max:22.8 \\
 $\sigma$:4.76 & min:1.6  \\
\hline
\end{tabular}      \\
 2021 & \begin{tabular}{ll}
\hline
 $\mu$:11.01   & max:19.5 \\
 $\sigma$:6.18 & min:0.1  \\
\hline
\end{tabular}      & \begin{tabular}{ll}
\hline
 $\mu$:8.94    & max:16.7 \\
 $\sigma$:4.72 & min:0.1  \\
\hline
\end{tabular}      & \begin{tabular}{ll}
\hline
 $\mu$:10.42   & max:21.4 \\
 $\sigma$:6.44 & min:-1.1 \\
\hline
\end{tabular}      & \begin{tabular}{ll}
\hline
 $\mu$:11.0   & max:23.2 \\
 $\sigma$:7.0 & min:-0.4 \\
\hline
\end{tabular}      \\
 2022 & \begin{tabular}{ll}
\hline
 $\mu$:10.62   & max:18.1 \\
 $\sigma$:6.13 & min:0.3  \\
\hline
\end{tabular}      & \begin{tabular}{ll}
\hline
 $\mu$:8.69    & max:16.4 \\
 $\sigma$:4.76 & min:0.6  \\
\hline
\end{tabular}      & \begin{tabular}{ll}
\hline
 $\mu$:10.31   & max:20.6 \\
 $\sigma$:6.31 & min:-1.8 \\
\hline
\end{tabular}      & \begin{tabular}{ll}
\hline
 $\mu$:10.84   & max:20.6 \\
 $\sigma$:5.97 & min:-0.1 \\
\hline
\end{tabular}      \\
\hline
\end{tabular}
	\caption[Table of station statistics for soil 10cm part 1]{The table shows the statistics of each station for each feature, except for Time as it is a strictly monotonic increasing sequence, for each year. The station names can be found in table \ref{tab:station:names}. $\mu$ Denotes the mean temperature, $\sigma$ denotes the standard deviation, "min" is the minimum temperature, and "max" is the maximum temperature. All values in the table have the unit degree Celsius.}
\end{table}
\begin{table}
	\begin{tabular}{rllll}
\hline
      & 38   & 34   & 27   & 42   \\
\hline
 2014 & \begin{tabular}{ll}
\hline
 $\mu$:12.52   & max:21.7 \\
 $\sigma$:5.42 & min:1.7  \\
\hline
\end{tabular}      & \begin{tabular}{ll}
\hline
 $\mu$:9.04    & max:16.7 \\
 $\sigma$:5.11 & min:-0.1 \\
\hline
\end{tabular}      & \begin{tabular}{ll}
\hline
 $\mu$:11.86   & max:21.4 \\
 $\sigma$:5.59 & min:0.9  \\
\hline
\end{tabular}      & \begin{tabular}{ll}
\hline
 $\mu$:12.97   & max:20.6 \\
 $\sigma$:4.05 & min:4.6  \\
\hline
\end{tabular}      \\
 2015 & \begin{tabular}{ll}
\hline
 $\mu$:10.85   & max:17.9 \\
 $\sigma$:5.14 & min:0.6  \\
\hline
\end{tabular}      & \begin{tabular}{ll}
\hline
 $\mu$:8.69    & max:14.6 \\
 $\sigma$:3.98 & min:0.8  \\
\hline
\end{tabular}      & \begin{tabular}{ll}
\hline
 $\mu$:10.2    & max:19.1 \\
 $\sigma$:5.33 & min:-0.1 \\
\hline
\end{tabular}      & \begin{tabular}{ll}
\hline
 $\mu$:10.77   & max:18.6 \\
 $\sigma$:5.04 & min:0.4  \\
\hline
\end{tabular}      \\
 2016 & \begin{tabular}{ll}
\hline
 $\mu$:11.56   & max:19.4 \\
 $\sigma$:5.92 & min:0.3  \\
\hline
\end{tabular}      & \begin{tabular}{ll}
\hline
 $\mu$:8.62    & max:15.7 \\
 $\sigma$:4.69 & min:0.0  \\
\hline
\end{tabular}      & \begin{tabular}{ll}
\hline
 $\mu$:10.21   & max:18.5 \\
 $\sigma$:6.22 & min:-2.2 \\
\hline
\end{tabular}      & \begin{tabular}{ll}
\hline
 $\mu$:11.11   & max:19.3 \\
 $\sigma$:5.78 & min:0.2  \\
\hline
\end{tabular}      \\
 2017 & \begin{tabular}{ll}
\hline
 $\mu$:10.37   & max:16.9 \\
 $\sigma$:5.43 & min:0.3  \\
\hline
\end{tabular}      & \begin{tabular}{ll}
\hline
 $\mu$:8.63    & max:15.7 \\
 $\sigma$:4.61 & min:0.4  \\
\hline
\end{tabular}      & \begin{tabular}{ll}
\hline
 $\mu$:9.49    & max:17.1 \\
 $\sigma$:5.87 & min:-1.3 \\
\hline
\end{tabular}      & \begin{tabular}{ll}
\hline
 $\mu$:10.32  & max:17.3 \\
 $\sigma$:5.5 & min:0.1  \\
\hline
\end{tabular}      \\
 2018 & \begin{tabular}{ll}
\hline
 $\mu$:11.09   & max:20.5 \\
 $\sigma$:6.15 & min:0.6  \\
\hline
\end{tabular}      & \begin{tabular}{ll}
\hline
 $\mu$:8.31    & max:16.0 \\
 $\sigma$:5.27 & min:-0.2 \\
\hline
\end{tabular}      & \begin{tabular}{ll}
\hline
 $\mu$:9.66    & max:19.9 \\
 $\sigma$:6.44 & min:-0.1 \\
\hline
\end{tabular}      & \begin{tabular}{ll}
\hline
 $\mu$:10.86   & max:19.9 \\
 $\sigma$:6.19 & min:0.3  \\
\hline
\end{tabular}      \\
 2019 & \begin{tabular}{ll}
\hline
 $\mu$:10.87   & max:20.4 \\
 $\sigma$:5.55 & min:0.8  \\
\hline
\end{tabular}      & \begin{tabular}{ll}
\hline
 $\mu$:9.03    & max:17.9 \\
 $\sigma$:4.91 & min:0.2  \\
\hline
\end{tabular}      & \begin{tabular}{ll}
\hline
 $\mu$:10.19   & max:20.7 \\
 $\sigma$:5.84 & min:0.3  \\
\hline
\end{tabular}      & \begin{tabular}{ll}
\hline
 $\mu$:10.52   & max:19.9 \\
 $\sigma$:5.77 & min:0.5  \\
\hline
\end{tabular}      \\
 2020 & \begin{tabular}{ll}
\hline
 $\mu$:9.76    & max:18.7 \\
 $\sigma$:5.75 & min:0.8  \\
\hline
\end{tabular}      & \begin{tabular}{ll}
\hline
 $\mu$:8.45    & max:15.6 \\
 $\sigma$:4.62 & min:0.6  \\
\hline
\end{tabular}      & \begin{tabular}{ll}
\hline
 $\mu$:10.21   & max:19.7 \\
 $\sigma$:5.83 & min:-0.5 \\
\hline
\end{tabular}      & \begin{tabular}{ll}
\hline
 $\mu$:12.79   & max:20.1 \\
 $\sigma$:4.49 & min:2.5  \\
\hline
\end{tabular}      \\
 2021 & \begin{tabular}{ll}
\hline
 $\mu$:10.77   & max:19.2 \\
 $\sigma$:6.01 & min:0.3  \\
\hline
\end{tabular}      & \begin{tabular}{ll}
\hline
 $\mu$:8.79    & max:15.7 \\
 $\sigma$:4.56 & min:0.2  \\
\hline
\end{tabular}      & \begin{tabular}{ll}
\hline
 $\mu$:10.03   & max:19.3 \\
 $\sigma$:6.19 & min:-0.9 \\
\hline
\end{tabular}      & \begin{tabular}{ll}
\hline
 $\mu$:10.75   & max:20.6 \\
 $\sigma$:6.74 & min:-0.1 \\
\hline
\end{tabular}      \\
 2022 & \begin{tabular}{ll}
\hline
 $\mu$:10.39  & max:17.6 \\
 $\sigma$:5.6 & min:0.5  \\
\hline
\end{tabular}      & \begin{tabular}{ll}
\hline
 $\mu$:8.61    & max:15.2 \\
 $\sigma$:4.62 & min:0.8  \\
\hline
\end{tabular}      & \begin{tabular}{ll}
\hline
 $\mu$:9.99    & max:18.5 \\
 $\sigma$:6.02 & min:-0.8 \\
\hline
\end{tabular}      & \begin{tabular}{ll}
\hline
 $\mu$:10.6    & max:18.5 \\
 $\sigma$:5.69 & min:0.3  \\
\hline
\end{tabular}      \\
\hline
\end{tabular}
	\caption[Table of station statistics for soil 20cm part 1]{The table shows the statistics of each station for each feature, except for Time as it is a strictly monotonic increasing sequence, for each year. The station names can be found in table \ref{tab:station:names}. $\mu$ Denotes the mean temperature, $\sigma$ denotes the standard deviation, "min" is the minimum temperature, and "max" is the maximum temperature. All values in the table have the unit degree Celsius.}
\end{table}

\begin{table}
	\begin{tabular}{rllll}
\hline
      & 30   & 41   & 39   & 50   \\
\hline
 2014 & \begin{tabular}{ll}
\hline
 $\mu$:11.49   & max:31.8 \\
 $\sigma$:6.47 & min:-3.3 \\
\hline
\end{tabular}      & \begin{tabular}{ll}
\hline
 $\mu$:12.2    & max:33.0 \\
 $\sigma$:6.63 & min:-4.5 \\
\hline
\end{tabular}      & \begin{tabular}{ll}
\hline
 $\mu$:10.76   & max:30.7 \\
 $\sigma$:6.24 & min:-8.0 \\
\hline
\end{tabular}      & \begin{tabular}{ll}
\hline
 $\mu$:11.87   & max:29.4 \\
 $\sigma$:5.69 & min:-1.9 \\
\hline
\end{tabular}      \\
 2015 & \begin{tabular}{ll}
\hline
 $\mu$:10.15   & max:25.7 \\
 $\sigma$:5.81 & min:-4.9 \\
\hline
\end{tabular}      & \begin{tabular}{ll}
\hline
 $\mu$:10.61   & max:26.4 \\
 $\sigma$:5.68 & min:-5.1 \\
\hline
\end{tabular}      & \begin{tabular}{ll}
\hline
 $\mu$:9.52    & max:26.2 \\
 $\sigma$:5.07 & min:-5.8 \\
\hline
\end{tabular}      & \begin{tabular}{ll}
\hline
 $\mu$:10.58   & max:24.9 \\
 $\sigma$:5.24 & min:-3.3 \\
\hline
\end{tabular}      \\
 2016 & \begin{tabular}{ll}
\hline
 $\mu$:12.43   & max:30.4 \\
 $\sigma$:6.28 & min:-4.4 \\
\hline
\end{tabular}      & \begin{tabular}{ll}
\hline
 $\mu$:11.22   & max:28.8 \\
 $\sigma$:6.41 & min:-5.5 \\
\hline
\end{tabular}      & \begin{tabular}{ll}
\hline
 $\mu$:9.51    & max:29.0 \\
 $\sigma$:5.45 & min:-6.4 \\
\hline
\end{tabular}      & \begin{tabular}{ll}
\hline
 $\mu$:11.35   & max:26.5 \\
 $\sigma$:6.02 & min:-4.9 \\
\hline
\end{tabular}      \\
 2017 & \begin{tabular}{ll}
\hline
 $\mu$:10.69   & max:26.7 \\
 $\sigma$:6.37 & min:-8.4 \\
\hline
\end{tabular}      & \begin{tabular}{ll}
\hline
 $\mu$:10.71   & max:27.3 \\
 $\sigma$:6.08 & min:-6.4 \\
\hline
\end{tabular}      & \begin{tabular}{ll}
\hline
 $\mu$:9.48    & max:26.2 \\
 $\sigma$:5.69 & min:-6.4 \\
\hline
\end{tabular}      & \begin{tabular}{ll}
\hline
 $\mu$:11.01   & max:24.7 \\
 $\sigma$:5.67 & min:-9.5 \\
\hline
\end{tabular}      \\
 2018 & \begin{tabular}{ll}
\hline
 $\mu$:11.32   & max:32.5  \\
 $\sigma$:9.11 & min:-16.8 \\
\hline
\end{tabular}      & \begin{tabular}{ll}
\hline
 $\mu$:11.55   & max:33.7  \\
 $\sigma$:8.63 & min:-19.2 \\
\hline
\end{tabular}      & \begin{tabular}{ll}
\hline
 $\mu$:9.25    & max:31.4  \\
 $\sigma$:6.88 & min:-15.1 \\
\hline
\end{tabular}      & \begin{tabular}{ll}
\hline
 $\mu$:11.73   & max:31.3  \\
 $\sigma$:7.94 & min:-14.0 \\
\hline
\end{tabular}      \\
 2019 & \begin{tabular}{ll}
\hline
 $\mu$:10.6    & max:32.0  \\
 $\sigma$:6.99 & min:-11.9 \\
\hline
\end{tabular}      & \begin{tabular}{ll}
\hline
 $\mu$:10.84   & max:30.4  \\
 $\sigma$:6.68 & min:-12.8 \\
\hline
\end{tabular}      & \begin{tabular}{ll}
\hline
 $\mu$:9.19    & max:30.8  \\
 $\sigma$:6.45 & min:-10.9 \\
\hline
\end{tabular}      & \begin{tabular}{ll}
\hline
 $\mu$:11.73   & max:29.5 \\
 $\sigma$:6.32 & min:-7.3 \\
\hline
\end{tabular}      \\
 2020 & \begin{tabular}{ll}
\hline
 $\mu$:11.02  & max:29.6 \\
 $\sigma$:6.7 & min:-7.2 \\
\hline
\end{tabular}      & \begin{tabular}{ll}
\hline
 $\mu$:11.32   & max:30.2 \\
 $\sigma$:6.32 & min:-6.6 \\
\hline
\end{tabular}      & \begin{tabular}{ll}
\hline
 $\mu$:9.44    & max:29.8 \\
 $\sigma$:6.22 & min:-5.3 \\
\hline
\end{tabular}      & \begin{tabular}{ll}
\hline
 $\mu$:11.44   & max:28.6 \\
 $\sigma$:5.97 & min:-5.3 \\
\hline
\end{tabular}      \\
 2021 & \begin{tabular}{ll}
\hline
 $\mu$:11.42   & max:30.5 \\
 $\sigma$:6.89 & min:-7.0 \\
\hline
\end{tabular}      & \begin{tabular}{ll}
\hline
 $\mu$:11.52   & max:29.8 \\
 $\sigma$:6.55 & min:-7.1 \\
\hline
\end{tabular}      & \begin{tabular}{ll}
\hline
 $\mu$:9.91   & max:30.0 \\
 $\sigma$:5.8 & min:-3.9 \\
\hline
\end{tabular}      & \begin{tabular}{ll}
\hline
 $\mu$:11.77  & max:26.9 \\
 $\sigma$:6.1 & min:-4.8 \\
\hline
\end{tabular}      \\
 2022 & \begin{tabular}{ll}
\hline
 $\mu$:11.13   & max:27.2 \\
 $\sigma$:6.84 & min:-7.6 \\
\hline
\end{tabular}      & \begin{tabular}{ll}
\hline
 $\mu$:11.19   & max:28.5 \\
 $\sigma$:6.63 & min:-7.5 \\
\hline
\end{tabular}      & \begin{tabular}{ll}
\hline
 $\mu$:9.5     & max:27.5 \\
 $\sigma$:5.25 & min:-6.0 \\
\hline
\end{tabular}      & \begin{tabular}{ll}
\hline
 $\mu$:11.5   & max:27.7 \\
 $\sigma$:6.0 & min:-5.0 \\
\hline
\end{tabular}      \\
\hline
\end{tabular}
	\caption[Table of station statistics for air part 2]{The table shows the statistics of each station for each feature, except for Time as it is a strictly monotonic increasing sequence, for each year. The station names can be found in table \ref{tab:station:names}. $\mu$ Denotes the mean temperature, $\sigma$ denotes the standard deviation, "min" is the minimum temperature, and "max" is the maximum temperature. All values in the table have the unit degree Celsius.}
\end{table}
\begin{table}
		\begin{tabular}{rllll}
\hline
      & 30   & 41   & 39   & 50   \\
\hline
 2014 & \begin{tabular}{ll}
\hline
 $\mu$:10.76   & max:19.2 \\
 $\sigma$:4.84 & min:1.6  \\
\hline
\end{tabular}      & \begin{tabular}{ll}
\hline
 $\mu$:13.38   & max:24.8 \\
 $\sigma$:5.72 & min:2.4  \\
\hline
\end{tabular}      & \begin{tabular}{ll}
\hline
 $\mu$:10.05   & max:19.6 \\
 $\sigma$:5.34 & min:0.0  \\
\hline
\end{tabular}      & \begin{tabular}{ll}
\hline
 $\mu$:11.19   & max:21.6 \\
 $\sigma$:5.04 & min:1.5  \\
\hline
\end{tabular}      \\
 2015 & \begin{tabular}{ll}
\hline
 $\mu$:9.37    & max:16.3 \\
 $\sigma$:4.52 & min:0.5  \\
\hline
\end{tabular}      & \begin{tabular}{ll}
\hline
 $\mu$:11.1    & max:20.6 \\
 $\sigma$:4.95 & min:0.8  \\
\hline
\end{tabular}      & \begin{tabular}{ll}
\hline
 $\mu$:9.12    & max:15.7 \\
 $\sigma$:4.11 & min:1.2  \\
\hline
\end{tabular}      & \begin{tabular}{ll}
\hline
 $\mu$:10.27   & max:19.5 \\
 $\sigma$:4.81 & min:0.2  \\
\hline
\end{tabular}      \\
 2016 & \begin{tabular}{ll}
\hline
 $\mu$:13.31  & max:23.9 \\
 $\sigma$:5.3 & min:-3.3 \\
\hline
\end{tabular}      & \begin{tabular}{ll}
\hline
 $\mu$:11.17   & max:21.9 \\
 $\sigma$:5.78 & min:-0.1 \\
\hline
\end{tabular}      & \begin{tabular}{ll}
\hline
 $\mu$:9.11    & max:16.9 \\
 $\sigma$:4.87 & min:0.1  \\
\hline
\end{tabular}      & \begin{tabular}{ll}
\hline
 $\mu$:11.61   & max:20.7 \\
 $\sigma$:5.56 & min:0.0  \\
\hline
\end{tabular}      \\
 2017 & \begin{tabular}{ll}
\hline
 $\mu$:11.17   & max:23.0 \\
 $\sigma$:5.97 & min:-0.2 \\
\hline
\end{tabular}      & \begin{tabular}{ll}
\hline
 $\mu$:11.0    & max:18.5 \\
 $\sigma$:4.96 & min:0.3  \\
\hline
\end{tabular}      & \begin{tabular}{ll}
\hline
 $\mu$:9.0     & max:16.9 \\
 $\sigma$:4.75 & min:0.3  \\
\hline
\end{tabular}      & \begin{tabular}{ll}
\hline
 $\mu$:10.9    & max:19.0 \\
 $\sigma$:4.83 & min:0.3  \\
\hline
\end{tabular}      \\
 2018 & \begin{tabular}{ll}
\hline
 $\mu$:11.53   & max:25.6 \\
 $\sigma$:6.86 & min:0.4  \\
\hline
\end{tabular}      & \begin{tabular}{ll}
\hline
 $\mu$:12.44   & max:25.9 \\
 $\sigma$:7.82 & min:-0.8 \\
\hline
\end{tabular}      & \begin{tabular}{ll}
\hline
 $\mu$:9.08    & max:19.2 \\
 $\sigma$:5.86 & min:-0.9 \\
\hline
\end{tabular}      & \begin{tabular}{ll}
\hline
 $\mu$:10.92   & max:21.9 \\
 $\sigma$:6.26 & min:-0.3 \\
\hline
\end{tabular}      \\
 2019 & \begin{tabular}{ll}
\hline
 $\mu$:11.16   & max:27.6 \\
 $\sigma$:6.38 & min:0.4  \\
\hline
\end{tabular}      & \begin{tabular}{ll}
\hline
 $\mu$:10.49  & max:20.3 \\
 $\sigma$:5.4 & min:0.2  \\
\hline
\end{tabular}      & \begin{tabular}{ll}
\hline
 $\mu$:9.39    & max:19.4 \\
 $\sigma$:4.91 & min:0.5  \\
\hline
\end{tabular}      & \begin{tabular}{ll}
\hline
 $\mu$:10.78   & max:21.0 \\
 $\sigma$:5.08 & min:0.3  \\
\hline
\end{tabular}      \\
 2020 & \begin{tabular}{ll}
\hline
 $\mu$:11.59   & max:26.1 \\
 $\sigma$:6.17 & min:0.1  \\
\hline
\end{tabular}      & \begin{tabular}{ll}
\hline
 $\mu$:12.21   & max:23.9 \\
 $\sigma$:6.33 & min:0.6  \\
\hline
\end{tabular}      & \begin{tabular}{ll}
\hline
 $\mu$:9.11    & max:17.6 \\
 $\sigma$:4.85 & min:1.0  \\
\hline
\end{tabular}      & \begin{tabular}{ll}
\hline
 $\mu$:11.33   & max:20.2 \\
 $\sigma$:4.97 & min:1.3  \\
\hline
\end{tabular}      \\
 2021 & \begin{tabular}{ll}
\hline
 $\mu$:11.46   & max:23.6 \\
 $\sigma$:6.65 & min:-0.9 \\
\hline
\end{tabular}      & \begin{tabular}{ll}
\hline
 $\mu$:11.89   & max:23.5 \\
 $\sigma$:6.34 & min:0.0  \\
\hline
\end{tabular}      & \begin{tabular}{ll}
\hline
 $\mu$:9.44    & max:16.9 \\
 $\sigma$:4.79 & min:0.5  \\
\hline
\end{tabular}      & \begin{tabular}{ll}
\hline
 $\mu$:11.51   & max:21.0 \\
 $\sigma$:5.81 & min:0.0  \\
\hline
\end{tabular}      \\
 2022 & \begin{tabular}{ll}
\hline
 $\mu$:11.19   & max:22.6 \\
 $\sigma$:5.96 & min:0.1  \\
\hline
\end{tabular}      & \begin{tabular}{ll}
\hline
 $\mu$:11.32   & max:21.8 \\
 $\sigma$:6.12 & min:-0.2 \\
\hline
\end{tabular}      & \begin{tabular}{ll}
\hline
 $\mu$:9.27    & max:15.9 \\
 $\sigma$:4.35 & min:1.0  \\
\hline
\end{tabular}      & \begin{tabular}{ll}
\hline
 $\mu$:11.47   & max:20.3 \\
 $\sigma$:5.39 & min:0.7  \\
\hline
\end{tabular}      \\
\hline
\end{tabular}
	\caption[Table of station statistics for soil 10cm part 2]{The table shows the statistics of each station for each feature, except for Time as it is a strictly monotonic increasing sequence, for each year. The station names can be found in table \ref{tab:station:names}. $\mu$ Denotes the mean temperature, $\sigma$ denotes the standard deviation, "min" is the minimum temperature, and "max" is the maximum temperature. All values in the table have the unit degree Celsius.}
\end{table}
\begin{table}
		\begin{tabular}{rllll}
\hline
      & 30   & 41   & 39   & 50   \\
\hline
 2014 & \begin{tabular}{ll}
\hline
 $\mu$:11.02   & max:18.9 \\
 $\sigma$:4.89 & min:1.6  \\
\hline
\end{tabular}      & \begin{tabular}{ll}
\hline
 $\mu$:12.91   & max:22.2 \\
 $\sigma$:5.34 & min:2.6  \\
\hline
\end{tabular}      & \begin{tabular}{ll}
\hline
 $\mu$:9.9     & max:18.4 \\
 $\sigma$:5.22 & min:0.0  \\
\hline
\end{tabular}      & \begin{tabular}{ll}
\hline
 $\mu$:10.51   & max:18.8 \\
 $\sigma$:4.77 & min:1.8  \\
\hline
\end{tabular}      \\
 2015 & \begin{tabular}{ll}
\hline
 $\mu$:9.66    & max:19.5 \\
 $\sigma$:4.49 & min:1.7  \\
\hline
\end{tabular}      & \begin{tabular}{ll}
\hline
 $\mu$:10.83   & max:18.2 \\
 $\sigma$:4.71 & min:1.4  \\
\hline
\end{tabular}      & \begin{tabular}{ll}
\hline
 $\mu$:9.02    & max:15.3 \\
 $\sigma$:4.02 & min:1.4  \\
\hline
\end{tabular}      & \begin{tabular}{ll}
\hline
 $\mu$:9.87    & max:17.9 \\
 $\sigma$:4.79 & min:0.4  \\
\hline
\end{tabular}      \\
 2016 & \begin{tabular}{ll}
\hline
 $\mu$:13.17   & max:21.1 \\
 $\sigma$:4.83 & min:-3.3 \\
\hline
\end{tabular}      & \begin{tabular}{ll}
\hline
 $\mu$:10.83   & max:19.7 \\
 $\sigma$:5.51 & min:0.0  \\
\hline
\end{tabular}      & \begin{tabular}{ll}
\hline
 $\mu$:9.01    & max:16.1 \\
 $\sigma$:4.76 & min:0.1  \\
\hline
\end{tabular}      & \begin{tabular}{ll}
\hline
 $\mu$:11.44   & max:19.2 \\
 $\sigma$:5.37 & min:0.3  \\
\hline
\end{tabular}      \\
 2017 & \begin{tabular}{ll}
\hline
 $\mu$:10.92   & max:19.8 \\
 $\sigma$:5.69 & min:-0.1 \\
\hline
\end{tabular}      & \begin{tabular}{ll}
\hline
 $\mu$:10.63   & max:16.4 \\
 $\sigma$:4.75 & min:0.4  \\
\hline
\end{tabular}      & \begin{tabular}{ll}
\hline
 $\mu$:8.92    & max:16.2 \\
 $\sigma$:4.64 & min:0.5  \\
\hline
\end{tabular}      & \begin{tabular}{ll}
\hline
 $\mu$:10.24   & max:17.5 \\
 $\sigma$:4.67 & min:0.5  \\
\hline
\end{tabular}      \\
 2018 & \begin{tabular}{ll}
\hline
 $\mu$:11.36  & max:22.3 \\
 $\sigma$:6.4 & min:0.6  \\
\hline
\end{tabular}      & \begin{tabular}{ll}
\hline
 $\mu$:11.76   & max:22.2 \\
 $\sigma$:7.33 & min:-0.3 \\
\hline
\end{tabular}      & \begin{tabular}{ll}
\hline
 $\mu$:8.93    & max:18.2 \\
 $\sigma$:5.75 & min:-0.3 \\
\hline
\end{tabular}      & \begin{tabular}{ll}
\hline
 $\mu$:10.59   & max:19.3 \\
 $\sigma$:5.96 & min:0.0  \\
\hline
\end{tabular}      \\
 2019 & \begin{tabular}{ll}
\hline
 $\mu$:11.03   & max:23.5 \\
 $\sigma$:5.87 & min:0.7  \\
\hline
\end{tabular}      & \begin{tabular}{ll}
\hline
 $\mu$:10.37   & max:18.5 \\
 $\sigma$:5.12 & min:0.7  \\
\hline
\end{tabular}      & \begin{tabular}{ll}
\hline
 $\mu$:9.31    & max:18.5 \\
 $\sigma$:4.77 & min:0.7  \\
\hline
\end{tabular}      & \begin{tabular}{ll}
\hline
 $\mu$:10.57  & max:19.3 \\
 $\sigma$:4.9 & min:0.5  \\
\hline
\end{tabular}      \\
 2020 & \begin{tabular}{ll}
\hline
 $\mu$:11.37   & max:22.5 \\
 $\sigma$:5.73 & min:0.4  \\
\hline
\end{tabular}      & \begin{tabular}{ll}
\hline
 $\mu$:11.8   & max:21.2 \\
 $\sigma$:6.0 & min:1.0  \\
\hline
\end{tabular}      & \begin{tabular}{ll}
\hline
 $\mu$:9.02    & max:16.3 \\
 $\sigma$:4.73 & min:1.1  \\
\hline
\end{tabular}      & \begin{tabular}{ll}
\hline
 $\mu$:11.05   & max:18.6 \\
 $\sigma$:4.78 & min:1.5  \\
\hline
\end{tabular}      \\
 2021 & \begin{tabular}{ll}
\hline
 $\mu$:11.12   & max:21.3 \\
 $\sigma$:6.41 & min:-0.3 \\
\hline
\end{tabular}      & \begin{tabular}{ll}
\hline
 $\mu$:11.7    & max:21.4 \\
 $\sigma$:6.08 & min:0.2  \\
\hline
\end{tabular}      & \begin{tabular}{ll}
\hline
 $\mu$:9.35    & max:16.4 \\
 $\sigma$:4.71 & min:0.6  \\
\hline
\end{tabular}      & \begin{tabular}{ll}
\hline
 $\mu$:11.05  & max:19.1 \\
 $\sigma$:5.6 & min:0.0  \\
\hline
\end{tabular}      \\
 2022 & \begin{tabular}{ll}
\hline
 $\mu$:10.98   & max:19.8 \\
 $\sigma$:5.68 & min:0.4  \\
\hline
\end{tabular}      & \begin{tabular}{ll}
\hline
 $\mu$:11.15   & max:19.5 \\
 $\sigma$:5.83 & min:0.1  \\
\hline
\end{tabular}      & \begin{tabular}{ll}
\hline
 $\mu$:9.15    & max:15.5 \\
 $\sigma$:4.27 & min:1.1  \\
\hline
\end{tabular}      & \begin{tabular}{ll}
\hline
 $\mu$:11.0    & max:18.5 \\
 $\sigma$:5.14 & min:0.8  \\
\hline
\end{tabular}      \\
\hline
\end{tabular}
	\caption[Table of station statistics for soil 20cm part 2]{The table shows the statistics of each station for each feature, except for Time as it is a strictly monotonic increasing sequence, for each year. The station names can be found in table \ref{tab:station:names}. $\mu$ Denotes the mean temperature, $\sigma$ denotes the standard deviation, "min" is the minimum temperature, and "max" is the maximum temperature. All values in the table have the unit degree Celsius.}
\end{table}

\begin{table}
		\begin{tabular}{rllll}
\hline
      & 26   & 57   & 118   & 37   \\
\hline
 2014 & \begin{tabular}{ll}
\hline
 $\mu$:10.33   & max:30.2 \\
 $\sigma$:6.83 & min:-5.1 \\
\hline
\end{tabular}      & \begin{tabular}{ll}
\hline
 $\mu$:10.92   & max:31.2  \\
 $\sigma$:6.59 & min:-11.9 \\
\hline
\end{tabular}      & \begin{tabular}{ll}
\hline
 $\mu$:11.79   & max:31.5 \\
 $\sigma$:6.38 & min:-3.5 \\
\hline
\end{tabular}       & \begin{tabular}{ll}
\hline
 $\mu$:10.8    & max:30.7 \\
 $\sigma$:6.44 & min:-4.6 \\
\hline
\end{tabular}      \\
 2015 & \begin{tabular}{ll}
\hline
 $\mu$:8.99    & max:27.1 \\
 $\sigma$:6.13 & min:-9.3 \\
\hline
\end{tabular}      & \begin{tabular}{ll}
\hline
 $\mu$:9.57    & max:27.5 \\
 $\sigma$:5.37 & min:-6.7 \\
\hline
\end{tabular}      & \begin{tabular}{ll}
\hline
 $\mu$:10.28   & max:24.6 \\
 $\sigma$:5.48 & min:-4.9 \\
\hline
\end{tabular}       & \begin{tabular}{ll}
\hline
 $\mu$:9.36    & max:26.2 \\
 $\sigma$:5.74 & min:-6.0 \\
\hline
\end{tabular}      \\
 2016 & \begin{tabular}{ll}
\hline
 $\mu$:9.74    & max:26.7 \\
 $\sigma$:6.73 & min:-7.2 \\
\hline
\end{tabular}      & \begin{tabular}{ll}
\hline
 $\mu$:9.54    & max:28.6 \\
 $\sigma$:6.02 & min:-8.1 \\
\hline
\end{tabular}      & \begin{tabular}{ll}
\hline
 $\mu$:10.99   & max:27.3 \\
 $\sigma$:6.22 & min:-5.8 \\
\hline
\end{tabular}       & \begin{tabular}{ll}
\hline
 $\mu$:10.16   & max:26.9 \\
 $\sigma$:6.44 & min:-9.8 \\
\hline
\end{tabular}      \\
 2017 & \begin{tabular}{ll}
\hline
 $\mu$:8.93    & max:28.5  \\
 $\sigma$:6.48 & min:-11.8 \\
\hline
\end{tabular}      & \begin{tabular}{ll}
\hline
 $\mu$:9.41   & max:27.1 \\
 $\sigma$:5.9 & min:-8.0 \\
\hline
\end{tabular}      & \begin{tabular}{ll}
\hline
 $\mu$:10.43  & max:25.2 \\
 $\sigma$:5.9 & min:-6.4 \\
\hline
\end{tabular}       & \begin{tabular}{ll}
\hline
 $\mu$:9.48    & max:24.6 \\
 $\sigma$:6.11 & min:-7.0 \\
\hline
\end{tabular}      \\
 2018 & \begin{tabular}{ll}
\hline
 $\mu$:9.95    & max:32.4  \\
 $\sigma$:10.0 & min:-26.3 \\
\hline
\end{tabular}      & \begin{tabular}{ll}
\hline
 $\mu$:9.59    & max:32.0  \\
 $\sigma$:7.61 & min:-21.4 \\
\hline
\end{tabular}      & \begin{tabular}{ll}
\hline
 $\mu$:11.34   & max:33.1  \\
 $\sigma$:8.46 & min:-16.6 \\
\hline
\end{tabular}       & \begin{tabular}{ll}
\hline
 $\mu$:10.51   & max:31.3  \\
 $\sigma$:8.76 & min:-20.1 \\
\hline
\end{tabular}      \\
 2019 & \begin{tabular}{ll}
\hline
 $\mu$:9.24    & max:30.4  \\
 $\sigma$:7.48 & min:-17.4 \\
\hline
\end{tabular}      & \begin{tabular}{ll}
\hline
 $\mu$:9.31    & max:32.4  \\
 $\sigma$:6.98 & min:-13.0 \\
\hline
\end{tabular}      & \begin{tabular}{ll}
\hline
 $\mu$:11.06   & max:31.5  \\
 $\sigma$:6.72 & min:-11.3 \\
\hline
\end{tabular}       & \begin{tabular}{ll}
\hline
 $\mu$:9.97    & max:31.6  \\
 $\sigma$:7.08 & min:-16.6 \\
\hline
\end{tabular}      \\
 2020 & \begin{tabular}{ll}
\hline
 $\mu$:9.95   & max:28.9 \\
 $\sigma$:6.9 & min:-8.5 \\
\hline
\end{tabular}      & \begin{tabular}{ll}
\hline
 $\mu$:9.62    & max:32.8 \\
 $\sigma$:6.62 & min:-6.2 \\
\hline
\end{tabular}      & \begin{tabular}{ll}
\hline
 $\mu$:11.44   & max:30.2 \\
 $\sigma$:6.45 & min:-6.6 \\
\hline
\end{tabular}       & \begin{tabular}{ll}
\hline
 $\mu$:10.4   & max:29.7 \\
 $\sigma$:6.7 & min:-7.4 \\
\hline
\end{tabular}      \\
 2021 & \begin{tabular}{ll}
\hline
 $\mu$:10.08   & max:28.1  \\
 $\sigma$:7.17 & min:-12.0 \\
\hline
\end{tabular}      & \begin{tabular}{ll}
\hline
 $\mu$:9.89    & max:29.9 \\
 $\sigma$:6.38 & min:-8.3 \\
\hline
\end{tabular}      & \begin{tabular}{ll}
\hline
 $\mu$:11.59   & max:29.4 \\
 $\sigma$:6.59 & min:-5.5 \\
\hline
\end{tabular}       & \begin{tabular}{ll}
\hline
 $\mu$:10.51   & max:28.2 \\
 $\sigma$:6.81 & min:-8.1 \\
\hline
\end{tabular}      \\
 2022 & \begin{tabular}{ll}
\hline
 $\mu$:9.56    & max:27.0  \\
 $\sigma$:7.24 & min:-12.8 \\
\hline
\end{tabular}      & \begin{tabular}{ll}
\hline
 $\mu$:9.72    & max:28.4 \\
 $\sigma$:5.79 & min:-8.2 \\
\hline
\end{tabular}      & \begin{tabular}{ll}
\hline
 $\mu$:11.4    & max:29.4 \\
 $\sigma$:6.71 & min:-7.3 \\
\hline
\end{tabular}       & \begin{tabular}{ll}
\hline
 $\mu$:10.16   & max:30.5 \\
 $\sigma$:6.92 & min:-9.7 \\
\hline
\end{tabular}      \\
\hline
\end{tabular}
	\caption[Table of station statistics for air part 3]{The table shows the statistics of each station for each feature, except for Time as it is a strictly monotonic increasing sequence, for each year. The station names can be found in table \ref{tab:station:names}. $\mu$ Denotes the mean temperature, $\sigma$ denotes the standard deviation, "min" is the minimum temperature, and "max" is the maximum temperature. All values in the table have the unit degree Celsius.}
\end{table}
\begin{table}
		\begin{tabular}{rllll}
\hline
      & 26   & 57   & 118   & 37   \\
\hline
 2014 & \begin{tabular}{ll}
\hline
 $\mu$:10.52   & max:21.7 \\
 $\sigma$:5.78 & min:-0.2 \\
\hline
\end{tabular}      & \begin{tabular}{ll}
\hline
 $\mu$:10.6   & max:22.2 \\
 $\sigma$:5.7 & min:0.0  \\
\hline
\end{tabular}      & \begin{tabular}{ll}
\hline
 $\mu$:11.9    & max:21.8 \\
 $\sigma$:5.26 & min:1.9  \\
\hline
\end{tabular}       & \begin{tabular}{ll}
\hline
 $\mu$:11.24   & max:21.6 \\
 $\sigma$:5.15 & min:2.0  \\
\hline
\end{tabular}      \\
 2015 & \begin{tabular}{ll}
\hline
 $\mu$:8.86    & max:18.2 \\
 $\sigma$:5.28 & min:-0.5 \\
\hline
\end{tabular}      & \begin{tabular}{ll}
\hline
 $\mu$:9.55    & max:16.7 \\
 $\sigma$:4.38 & min:0.1  \\
\hline
\end{tabular}      & \begin{tabular}{ll}
\hline
 $\mu$:10.51   & max:19.5 \\
 $\sigma$:5.02 & min:0.5  \\
\hline
\end{tabular}       & \begin{tabular}{ll}
\hline
 $\mu$:9.76    & max:17.8 \\
 $\sigma$:4.51 & min:0.5  \\
\hline
\end{tabular}      \\
 2016 & \begin{tabular}{ll}
\hline
 $\mu$:9.26    & max:19.1 \\
 $\sigma$:6.05 & min:-2.3 \\
\hline
\end{tabular}      & \begin{tabular}{ll}
\hline
 $\mu$:9.15    & max:17.4 \\
 $\sigma$:5.25 & min:-0.2 \\
\hline
\end{tabular}      & \begin{tabular}{ll}
\hline
 $\mu$:10.87  & max:19.6 \\
 $\sigma$:5.6 & min:-0.4 \\
\hline
\end{tabular}       & \begin{tabular}{ll}
\hline
 $\mu$:10.59   & max:19.8 \\
 $\sigma$:5.56 & min:-0.4 \\
\hline
\end{tabular}      \\
 2017 & \begin{tabular}{ll}
\hline
 $\mu$:8.9     & max:18.5 \\
 $\sigma$:6.11 & min:-2.7 \\
\hline
\end{tabular}      & \begin{tabular}{ll}
\hline
 $\mu$:9.04    & max:16.3 \\
 $\sigma$:4.87 & min:-0.1 \\
\hline
\end{tabular}      & \begin{tabular}{ll}
\hline
 $\mu$:10.1    & max:16.9 \\
 $\sigma$:4.97 & min:-0.2 \\
\hline
\end{tabular}       & \begin{tabular}{ll}
\hline
 $\mu$:10.15   & max:18.5 \\
 $\sigma$:5.15 & min:-0.2 \\
\hline
\end{tabular}      \\
 2018 & \begin{tabular}{ll}
\hline
 $\mu$:9.85    & max:22.6 \\
 $\sigma$:7.19 & min:-1.4 \\
\hline
\end{tabular}      & \begin{tabular}{ll}
\hline
 $\mu$:8.46   & max:17.1 \\
 $\sigma$:5.6 & min:-1.3 \\
\hline
\end{tabular}      & \begin{tabular}{ll}
\hline
 $\mu$:10.15   & max:19.4 \\
 $\sigma$:6.35 & min:-0.9 \\
\hline
\end{tabular}       & \begin{tabular}{ll}
\hline
 $\mu$:11.1    & max:21.2 \\
 $\sigma$:6.22 & min:-1.5 \\
\hline
\end{tabular}      \\
 2019 & \begin{tabular}{ll}
\hline
 $\mu$:10.17   & max:21.4 \\
 $\sigma$:5.66 & min:0.0  \\
\hline
\end{tabular}      & \begin{tabular}{ll}
\hline
 $\mu$:9.2     & max:19.4 \\
 $\sigma$:5.27 & min:0.1  \\
\hline
\end{tabular}      & \begin{tabular}{ll}
\hline
 $\mu$:10.67   & max:20.9 \\
 $\sigma$:5.49 & min:0.1  \\
\hline
\end{tabular}       & \begin{tabular}{ll}
\hline
 $\mu$:10.77   & max:21.4 \\
 $\sigma$:5.61 & min:-0.2 \\
\hline
\end{tabular}      \\
 2020 & \begin{tabular}{ll}
\hline
 $\mu$:9.52    & max:22.5 \\
 $\sigma$:6.26 & min:-2.5 \\
\hline
\end{tabular}      & \begin{tabular}{ll}
\hline
 $\mu$:9.55    & max:18.0 \\
 $\sigma$:5.29 & min:0.1  \\
\hline
\end{tabular}      & \begin{tabular}{ll}
\hline
 $\mu$:10.82   & max:19.6 \\
 $\sigma$:5.07 & min:0.4  \\
\hline
\end{tabular}       & \begin{tabular}{ll}
\hline
 $\mu$:10.81   & max:21.0 \\
 $\sigma$:5.29 & min:0.0  \\
\hline
\end{tabular}      \\
 2021 & \begin{tabular}{ll}
\hline
 $\mu$:9.59    & max:20.9 \\
 $\sigma$:6.59 & min:-3.0 \\
\hline
\end{tabular}      & \begin{tabular}{ll}
\hline
 $\mu$:9.21    & max:17.8 \\
 $\sigma$:5.32 & min:-0.3 \\
\hline
\end{tabular}      & \begin{tabular}{ll}
\hline
 $\mu$:10.28   & max:19.6 \\
 $\sigma$:5.91 & min:-0.7 \\
\hline
\end{tabular}       & \begin{tabular}{ll}
\hline
 $\mu$:10.59  & max:22.0 \\
 $\sigma$:6.4 & min:-0.5 \\
\hline
\end{tabular}      \\
 2022 & \begin{tabular}{ll}
\hline
 $\mu$:9.1     & max:20.4 \\
 $\sigma$:6.18 & min:-2.4 \\
\hline
\end{tabular}      & \begin{tabular}{ll}
\hline
 $\mu$:9.71    & max:18.1 \\
 $\sigma$:4.97 & min:0.3  \\
\hline
\end{tabular}      & \begin{tabular}{ll}
\hline
 $\mu$:10.22   & max:17.4 \\
 $\sigma$:5.15 & min:0.2  \\
\hline
\end{tabular}       & \begin{tabular}{ll}
\hline
 $\mu$:10.34   & max:19.9 \\
 $\sigma$:5.99 & min:-0.3 \\
\hline
\end{tabular}      \\
\hline
\end{tabular}
	\caption[Table of station statistics for soil 10cm part 3]{The table shows the statistics of each station for each feature, except for Time as it is a strictly monotonic increasing sequence, for each year. The station names can be found in table \ref{tab:station:names}. $\mu$ Denotes the mean temperature, $\sigma$ denotes the standard deviation, "min" is the minimum temperature, and "max" is the maximum temperature. All values in the table have the unit degree Celsius.}
\end{table}
\begin{table}
		\begin{tabular}{rllll}
\hline
      & 26   & 57   & 118   & 37   \\
\hline
 2014 & \begin{tabular}{ll}
\hline
 $\mu$:10.49   & max:19.7 \\
 $\sigma$:5.59 & min:0.1  \\
\hline
\end{tabular}      & \begin{tabular}{ll}
\hline
 $\mu$:10.15   & max:19.1 \\
 $\sigma$:5.36 & min:-0.1 \\
\hline
\end{tabular}      & \begin{tabular}{ll}
\hline
 $\mu$:11.67  & max:20.3 \\
 $\sigma$:5.0 & min:2.4  \\
\hline
\end{tabular}       & \begin{tabular}{ll}
\hline
 $\mu$:11.15   & max:19.7 \\
 $\sigma$:4.94 & min:2.3  \\
\hline
\end{tabular}      \\
 2015 & \begin{tabular}{ll}
\hline
 $\mu$:9.01    & max:17.5 \\
 $\sigma$:5.33 & min:-0.2 \\
\hline
\end{tabular}      & \begin{tabular}{ll}
\hline
 $\mu$:9.34    & max:15.5 \\
 $\sigma$:4.32 & min:0.2  \\
\hline
\end{tabular}      & \begin{tabular}{ll}
\hline
 $\mu$:10.4    & max:18.0 \\
 $\sigma$:4.73 & min:0.9  \\
\hline
\end{tabular}       & \begin{tabular}{ll}
\hline
 $\mu$:9.72    & max:16.6 \\
 $\sigma$:4.53 & min:0.6  \\
\hline
\end{tabular}      \\
 2016 & \begin{tabular}{ll}
\hline
 $\mu$:9.59    & max:18.4 \\
 $\sigma$:6.09 & min:-1.6 \\
\hline
\end{tabular}      & \begin{tabular}{ll}
\hline
 $\mu$:8.95    & max:16.2 \\
 $\sigma$:5.12 & min:0.0  \\
\hline
\end{tabular}      & \begin{tabular}{ll}
\hline
 $\mu$:10.77   & max:18.3 \\
 $\sigma$:5.26 & min:-0.1 \\
\hline
\end{tabular}       & \begin{tabular}{ll}
\hline
 $\mu$:10.39   & max:18.3 \\
 $\sigma$:5.42 & min:-0.4 \\
\hline
\end{tabular}      \\
 2017 & \begin{tabular}{ll}
\hline
 $\mu$:9.1     & max:17.7 \\
 $\sigma$:6.18 & min:-1.8 \\
\hline
\end{tabular}      & \begin{tabular}{ll}
\hline
 $\mu$:8.8     & max:15.2 \\
 $\sigma$:4.74 & min:0.1  \\
\hline
\end{tabular}      & \begin{tabular}{ll}
\hline
 $\mu$:9.99    & max:16.0 \\
 $\sigma$:4.77 & min:0.2  \\
\hline
\end{tabular}       & \begin{tabular}{ll}
\hline
 $\mu$:10.12   & max:16.9 \\
 $\sigma$:5.04 & min:0.0  \\
\hline
\end{tabular}      \\
 2018 & \begin{tabular}{ll}
\hline
 $\mu$:10.14   & max:20.6 \\
 $\sigma$:7.06 & min:-0.8 \\
\hline
\end{tabular}      & \begin{tabular}{ll}
\hline
 $\mu$:8.11    & max:15.6 \\
 $\sigma$:5.41 & min:-0.9 \\
\hline
\end{tabular}      & \begin{tabular}{ll}
\hline
 $\mu$:9.93    & max:18.3 \\
 $\sigma$:6.07 & min:-0.6 \\
\hline
\end{tabular}       & \begin{tabular}{ll}
\hline
 $\mu$:10.89   & max:20.0 \\
 $\sigma$:6.13 & min:-0.7 \\
\hline
\end{tabular}      \\
 2019 & \begin{tabular}{ll}
\hline
 $\mu$:9.87    & max:19.9 \\
 $\sigma$:5.89 & min:0.3  \\
\hline
\end{tabular}      & \begin{tabular}{ll}
\hline
 $\mu$:9.02    & max:18.3 \\
 $\sigma$:5.12 & min:0.2  \\
\hline
\end{tabular}      & \begin{tabular}{ll}
\hline
 $\mu$:10.45   & max:19.4 \\
 $\sigma$:5.27 & min:0.3  \\
\hline
\end{tabular}       & \begin{tabular}{ll}
\hline
 $\mu$:10.46   & max:19.3 \\
 $\sigma$:5.43 & min:-0.3 \\
\hline
\end{tabular}      \\
 2020 & \begin{tabular}{ll}
\hline
 $\mu$:9.24    & max:20.1 \\
 $\sigma$:6.06 & min:-2.2 \\
\hline
\end{tabular}      & \begin{tabular}{ll}
\hline
 $\mu$:9.37    & max:16.8 \\
 $\sigma$:5.17 & min:0.1  \\
\hline
\end{tabular}      & \begin{tabular}{ll}
\hline
 $\mu$:10.61   & max:18.3 \\
 $\sigma$:4.84 & min:0.8  \\
\hline
\end{tabular}       & \begin{tabular}{ll}
\hline
 $\mu$:10.83   & max:19.4 \\
 $\sigma$:5.16 & min:0.5  \\
\hline
\end{tabular}      \\
 2021 & \begin{tabular}{ll}
\hline
 $\mu$:9.35    & max:19.3 \\
 $\sigma$:6.37 & min:-2.0 \\
\hline
\end{tabular}      & \begin{tabular}{ll}
\hline
 $\mu$:8.96    & max:16.6 \\
 $\sigma$:5.18 & min:-0.3 \\
\hline
\end{tabular}      & \begin{tabular}{ll}
\hline
 $\mu$:10.0    & max:18.1 \\
 $\sigma$:5.67 & min:-0.7 \\
\hline
\end{tabular}       & \begin{tabular}{ll}
\hline
 $\mu$:10.52   & max:20.5 \\
 $\sigma$:6.31 & min:-0.3 \\
\hline
\end{tabular}      \\
 2022 & \begin{tabular}{ll}
\hline
 $\mu$:8.85   & max:18.5 \\
 $\sigma$:6.0 & min:-1.7 \\
\hline
\end{tabular}      & \begin{tabular}{ll}
\hline
 $\mu$:9.48    & max:16.8 \\
 $\sigma$:4.84 & min:0.4  \\
\hline
\end{tabular}      & \begin{tabular}{ll}
\hline
 $\mu$:9.93    & max:16.5 \\
 $\sigma$:4.95 & min:0.5  \\
\hline
\end{tabular}       & \begin{tabular}{ll}
\hline
 $\mu$:10.38   & max:18.7 \\
 $\sigma$:5.87 & min:0.0  \\
\hline
\end{tabular}      \\
\hline
\end{tabular}
	\caption[Table of station statistics for soil 20cm part 3]{The table shows the statistics of each station for each feature, except for Time as it is a strictly monotonic increasing sequence, for each year. The station names can be found in table \ref{tab:station:names}. $\mu$ Denotes the mean temperature, $\sigma$ denotes the standard deviation, "min" is the minimum temperature, and "max" is the maximum temperature. All values in the table have the unit degree Celsius.}
\end{table}

\begin{table}
		\begin{tabular}{rllll}
\hline
      & 52   & 11   & 15   & 17   \\
\hline
 2014 & \begin{tabular}{ll}
\hline
 $\mu$:11.96   & max:31.9 \\
 $\sigma$:6.68 & min:-4.0 \\
\hline
\end{tabular}      & \begin{tabular}{ll}
\hline
 $\mu$:10.66   & max:30.0 \\
 $\sigma$:6.47 & min:-4.5 \\
\hline
\end{tabular}      & \begin{tabular}{ll}
\hline
 $\mu$:11.17  & max:32.4 \\
 $\sigma$:6.3 & min:-9.4 \\
\hline
\end{tabular}      & \begin{tabular}{ll}
\hline
 $\mu$:nan    & max:nan \\
 $\sigma$:nan & min:nan \\
\hline
\end{tabular}      \\
 2015 & \begin{tabular}{ll}
\hline
 $\mu$:10.45   & max:25.8 \\
 $\sigma$:5.93 & min:-6.4 \\
\hline
\end{tabular}      & \begin{tabular}{ll}
\hline
 $\mu$:9.12    & max:28.6 \\
 $\sigma$:5.76 & min:-7.1 \\
\hline
\end{tabular}      & \begin{tabular}{ll}
\hline
 $\mu$:9.86    & max:27.9 \\
 $\sigma$:5.17 & min:-2.9 \\
\hline
\end{tabular}      & \begin{tabular}{ll}
\hline
 $\mu$:nan    & max:nan \\
 $\sigma$:nan & min:nan \\
\hline
\end{tabular}      \\
 2016 & \begin{tabular}{ll}
\hline
 $\mu$:11.1    & max:28.4 \\
 $\sigma$:6.46 & min:-5.7 \\
\hline
\end{tabular}      & \begin{tabular}{ll}
\hline
 $\mu$:9.84    & max:26.4 \\
 $\sigma$:6.49 & min:-7.4 \\
\hline
\end{tabular}      & \begin{tabular}{ll}
\hline
 $\mu$:9.98    & max:28.1 \\
 $\sigma$:5.58 & min:-4.2 \\
\hline
\end{tabular}      & \begin{tabular}{ll}
\hline
 $\mu$:11.34   & max:25.8 \\
 $\sigma$:6.03 & min:-5.5 \\
\hline
\end{tabular}      \\
 2017 & \begin{tabular}{ll}
\hline
 $\mu$:10.37   & max:25.6 \\
 $\sigma$:6.16 & min:-6.7 \\
\hline
\end{tabular}      & \begin{tabular}{ll}
\hline
 $\mu$:9.05    & max:28.5 \\
 $\sigma$:6.17 & min:-8.7 \\
\hline
\end{tabular}      & \begin{tabular}{ll}
\hline
 $\mu$:9.67   & max:26.5 \\
 $\sigma$:5.6 & min:-6.1 \\
\hline
\end{tabular}      & \begin{tabular}{ll}
\hline
 $\mu$:8.24    & max:25.2  \\
 $\sigma$:6.65 & min:-16.7 \\
\hline
\end{tabular}      \\
 2018 & \begin{tabular}{ll}
\hline
 $\mu$:11.45   & max:32.6  \\
 $\sigma$:8.54 & min:-18.0 \\
\hline
\end{tabular}      & \begin{tabular}{ll}
\hline
 $\mu$:10.17   & max:30.2  \\
 $\sigma$:9.32 & min:-21.4 \\
\hline
\end{tabular}      & \begin{tabular}{ll}
\hline
 $\mu$:9.88    & max:32.6  \\
 $\sigma$:7.04 & min:-15.4 \\
\hline
\end{tabular}      & \begin{tabular}{ll}
\hline
 $\mu$:9.07    & max:29.8  \\
 $\sigma$:9.79 & min:-25.9 \\
\hline
\end{tabular}      \\
 2019 & \begin{tabular}{ll}
\hline
 $\mu$:10.7    & max:30.6  \\
 $\sigma$:6.75 & min:-13.2 \\
\hline
\end{tabular}      & \begin{tabular}{ll}
\hline
 $\mu$:9.28    & max:29.4  \\
 $\sigma$:7.08 & min:-13.2 \\
\hline
\end{tabular}      & \begin{tabular}{ll}
\hline
 $\mu$:9.7     & max:33.5  \\
 $\sigma$:6.67 & min:-10.0 \\
\hline
\end{tabular}      & \begin{tabular}{ll}
\hline
 $\mu$:8.75    & max:29.9  \\
 $\sigma$:7.46 & min:-20.4 \\
\hline
\end{tabular}      \\
 2020 & \begin{tabular}{ll}
\hline
 $\mu$:11.07  & max:28.9 \\
 $\sigma$:6.5 & min:-7.1 \\
\hline
\end{tabular}      & \begin{tabular}{ll}
\hline
 $\mu$:9.92    & max:28.5 \\
 $\sigma$:6.66 & min:-9.0 \\
\hline
\end{tabular}      & \begin{tabular}{ll}
\hline
 $\mu$:9.94    & max:32.5 \\
 $\sigma$:6.46 & min:-4.5 \\
\hline
\end{tabular}      & \begin{tabular}{ll}
\hline
 $\mu$:9.64    & max:29.4  \\
 $\sigma$:7.22 & min:-18.3 \\
\hline
\end{tabular}      \\
 2021 & \begin{tabular}{ll}
\hline
 $\mu$:11.06   & max:27.7 \\
 $\sigma$:6.62 & min:-7.6 \\
\hline
\end{tabular}      & \begin{tabular}{ll}
\hline
 $\mu$:10.18   & max:27.7 \\
 $\sigma$:6.76 & min:-8.8 \\
\hline
\end{tabular}      & \begin{tabular}{ll}
\hline
 $\mu$:10.38   & max:30.6 \\
 $\sigma$:6.12 & min:-4.8 \\
\hline
\end{tabular}      & \begin{tabular}{ll}
\hline
 $\mu$:nan    & max:nan \\
 $\sigma$:nan & min:nan \\
\hline
\end{tabular}      \\
 2022 & \begin{tabular}{ll}
\hline
 $\mu$:10.88   & max:28.0 \\
 $\sigma$:6.85 & min:-8.7 \\
\hline
\end{tabular}      & \begin{tabular}{ll}
\hline
 $\mu$:9.64    & max:26.8  \\
 $\sigma$:6.65 & min:-10.3 \\
\hline
\end{tabular}      & \begin{tabular}{ll}
\hline
 $\mu$:10.03  & max:30.2 \\
 $\sigma$:5.5 & min:-4.8 \\
\hline
\end{tabular}      & \begin{tabular}{ll}
\hline
 $\mu$:9.07    & max:27.7  \\
 $\sigma$:7.16 & min:-15.0 \\
\hline
\end{tabular}      \\
\hline
\end{tabular}
	\caption[Table of station statistics for air part 4]{The table shows the statistics of each station for each feature, except for Time as it is a strictly monotonic increasing sequence, for each year. The station names can be found in table \ref{tab:station:names}. $\mu$ Denotes the mean temperature, $\sigma$ denotes the standard deviation, "min" is the minimum temperature, and "max" is the maximum temperature. All values in the table have the unit degree Celsius.}
\end{table}
\begin{table}
		\begin{tabular}{rllll}
\hline
      & 52   & 11   & 15   & 17   \\
\hline
 2014 & \begin{tabular}{ll}
\hline
 $\mu$:12.46   & max:23.2 \\
 $\sigma$:5.16 & min:2.6  \\
\hline
\end{tabular}      & \begin{tabular}{ll}
\hline
 $\mu$:11.15   & max:23.3 \\
 $\sigma$:5.98 & min:0.1  \\
\hline
\end{tabular}      & \begin{tabular}{ll}
\hline
 $\mu$:9.38    & max:17.2 \\
 $\sigma$:4.48 & min:0.5  \\
\hline
\end{tabular}      & \begin{tabular}{ll}
\hline
 $\mu$:nan    & max:nan \\
 $\sigma$:nan & min:nan \\
\hline
\end{tabular}      \\
 2015 & \begin{tabular}{ll}
\hline
 $\mu$:11.31   & max:19.4 \\
 $\sigma$:4.86 & min:1.1  \\
\hline
\end{tabular}      & \begin{tabular}{ll}
\hline
 $\mu$:8.04    & max:19.7 \\
 $\sigma$:5.15 & min:-0.2 \\
\hline
\end{tabular}      & \begin{tabular}{ll}
\hline
 $\mu$:8.81    & max:15.6 \\
 $\sigma$:3.98 & min:1.5  \\
\hline
\end{tabular}      & \begin{tabular}{ll}
\hline
 $\mu$:nan    & max:nan \\
 $\sigma$:nan & min:nan \\
\hline
\end{tabular}      \\
 2016 & \begin{tabular}{ll}
\hline
 $\mu$:11.95   & max:20.2 \\
 $\sigma$:5.67 & min:0.1  \\
\hline
\end{tabular}      & \begin{tabular}{ll}
\hline
 $\mu$:10.03   & max:20.0 \\
 $\sigma$:6.29 & min:-1.3 \\
\hline
\end{tabular}      & \begin{tabular}{ll}
\hline
 $\mu$:8.88   & max:17.3 \\
 $\sigma$:4.8 & min:-0.2 \\
\hline
\end{tabular}      & \begin{tabular}{ll}
\hline
 $\mu$:12.88   & max:20.7 \\
 $\sigma$:4.55 & min:2.5  \\
\hline
\end{tabular}      \\
 2017 & \begin{tabular}{ll}
\hline
 $\mu$:10.89   & max:19.7 \\
 $\sigma$:5.34 & min:0.1  \\
\hline
\end{tabular}      & \begin{tabular}{ll}
\hline
 $\mu$:9.31    & max:18.3 \\
 $\sigma$:6.13 & min:-1.4 \\
\hline
\end{tabular}      & \begin{tabular}{ll}
\hline
 $\mu$:8.93    & max:16.3 \\
 $\sigma$:4.52 & min:-0.1 \\
\hline
\end{tabular}      & \begin{tabular}{ll}
\hline
 $\mu$:8.66    & max:19.7 \\
 $\sigma$:6.31 & min:-1.9 \\
\hline
\end{tabular}      \\
 2018 & \begin{tabular}{ll}
\hline
 $\mu$:11.39  & max:21.2 \\
 $\sigma$:6.8 & min:-1.0 \\
\hline
\end{tabular}      & \begin{tabular}{ll}
\hline
 $\mu$:10.31   & max:21.9 \\
 $\sigma$:6.95 & min:0.0  \\
\hline
\end{tabular}      & \begin{tabular}{ll}
\hline
 $\mu$:8.82    & max:18.5 \\
 $\sigma$:5.22 & min:-0.1 \\
\hline
\end{tabular}      & \begin{tabular}{ll}
\hline
 $\mu$:9.55    & max:22.8 \\
 $\sigma$:6.99 & min:-0.5 \\
\hline
\end{tabular}      \\
 2019 & \begin{tabular}{ll}
\hline
 $\mu$:11.66  & max:21.7 \\
 $\sigma$:5.5 & min:0.3  \\
\hline
\end{tabular}      & \begin{tabular}{ll}
\hline
 $\mu$:9.73    & max:22.0 \\
 $\sigma$:6.26 & min:-0.1 \\
\hline
\end{tabular}      & \begin{tabular}{ll}
\hline
 $\mu$:9.24    & max:20.2 \\
 $\sigma$:4.97 & min:0.2  \\
\hline
\end{tabular}      & \begin{tabular}{ll}
\hline
 $\mu$:9.09    & max:21.4 \\
 $\sigma$:6.32 & min:-0.5 \\
\hline
\end{tabular}      \\
 2020 & \begin{tabular}{ll}
\hline
 $\mu$:12.33   & max:21.7 \\
 $\sigma$:4.97 & min:0.8  \\
\hline
\end{tabular}      & \begin{tabular}{ll}
\hline
 $\mu$:9.93    & max:22.4 \\
 $\sigma$:6.39 & min:-1.6 \\
\hline
\end{tabular}      & \begin{tabular}{ll}
\hline
 $\mu$:9.29    & max:19.6 \\
 $\sigma$:5.11 & min:0.2  \\
\hline
\end{tabular}      & \begin{tabular}{ll}
\hline
 $\mu$:9.71    & max:20.8 \\
 $\sigma$:6.53 & min:-1.2 \\
\hline
\end{tabular}      \\
 2021 & \begin{tabular}{ll}
\hline
 $\mu$:6.95    & max:22.1 \\
 $\sigma$:5.75 & min:0.0  \\
\hline
\end{tabular}      & \begin{tabular}{ll}
\hline
 $\mu$:10.11   & max:20.4 \\
 $\sigma$:6.25 & min:-0.3 \\
\hline
\end{tabular}      & \begin{tabular}{ll}
\hline
 $\mu$:9.9    & max:19.1 \\
 $\sigma$:5.1 & min:0.4  \\
\hline
\end{tabular}      & \begin{tabular}{ll}
\hline
 $\mu$:nan    & max:nan \\
 $\sigma$:nan & min:nan \\
\hline
\end{tabular}      \\
 2022 & \begin{tabular}{ll}
\hline
 $\mu$:14.0    & max:19.4 \\
 $\sigma$:3.07 & min:7.4  \\
\hline
\end{tabular}      & \begin{tabular}{ll}
\hline
 $\mu$:13.18   & max:21.3 \\
 $\sigma$:3.95 & min:3.9  \\
\hline
\end{tabular}      & \begin{tabular}{ll}
\hline
 $\mu$:9.42    & max:17.1 \\
 $\sigma$:4.41 & min:0.4  \\
\hline
\end{tabular}      & \begin{tabular}{ll}
\hline
 $\mu$:9.33   & max:19.9 \\
 $\sigma$:6.4 & min:-2.4 \\
\hline
\end{tabular}      \\
\hline
\end{tabular}
	\caption[Table of station statistics for soil 10cm part 4]{The table shows the statistics of each station for each feature, except for Time as it is a strictly monotonic increasing sequence, for each year. The station names can be found in table \ref{tab:station:names}. $\mu$ Denotes the mean temperature, $\sigma$ denotes the standard deviation, "min" is the minimum temperature, and "max" is the maximum temperature. All values in the table have the unit degree Celsius.}
\end{table}
\begin{table}
		\begin{tabular}{rllll}
\hline
      & 52   & 11   & 15   & 17   \\
\hline
 2014 & \begin{tabular}{ll}
\hline
 $\mu$:12.15   & max:20.9 \\
 $\sigma$:4.93 & min:3.3  \\
\hline
\end{tabular}      & \begin{tabular}{ll}
\hline
 $\mu$:10.92   & max:21.6 \\
 $\sigma$:5.76 & min:0.3  \\
\hline
\end{tabular}      & \begin{tabular}{ll}
\hline
 $\mu$:9.21    & max:16.0 \\
 $\sigma$:4.35 & min:0.5  \\
\hline
\end{tabular}      & \begin{tabular}{ll}
\hline
 $\mu$:nan    & max:nan \\
 $\sigma$:nan & min:nan \\
\hline
\end{tabular}      \\
 2015 & \begin{tabular}{ll}
\hline
 $\mu$:11.08  & max:17.8 \\
 $\sigma$:4.7 & min:1.6  \\
\hline
\end{tabular}      & \begin{tabular}{ll}
\hline
 $\mu$:7.92    & max:17.9 \\
 $\sigma$:4.91 & min:0.0  \\
\hline
\end{tabular}      & \begin{tabular}{ll}
\hline
 $\mu$:8.71    & max:15.0 \\
 $\sigma$:3.88 & min:1.7  \\
\hline
\end{tabular}      & \begin{tabular}{ll}
\hline
 $\mu$:nan    & max:nan \\
 $\sigma$:nan & min:nan \\
\hline
\end{tabular}      \\
 2016 & \begin{tabular}{ll}
\hline
 $\mu$:11.67   & max:18.7 \\
 $\sigma$:5.44 & min:0.4  \\
\hline
\end{tabular}      & \begin{tabular}{ll}
\hline
 $\mu$:9.84    & max:18.5 \\
 $\sigma$:6.06 & min:-0.7 \\
\hline
\end{tabular}      & \begin{tabular}{ll}
\hline
 $\mu$:8.75    & max:16.1 \\
 $\sigma$:4.63 & min:0.0  \\
\hline
\end{tabular}      & \begin{tabular}{ll}
\hline
 $\mu$:12.63  & max:18.0 \\
 $\sigma$:4.0 & min:3.6  \\
\hline
\end{tabular}      \\
 2017 & \begin{tabular}{ll}
\hline
 $\mu$:10.64   & max:18.0 \\
 $\sigma$:5.15 & min:0.2  \\
\hline
\end{tabular}      & \begin{tabular}{ll}
\hline
 $\mu$:9.1     & max:16.8 \\
 $\sigma$:5.99 & min:-1.1 \\
\hline
\end{tabular}      & \begin{tabular}{ll}
\hline
 $\mu$:8.8     & max:15.4 \\
 $\sigma$:4.38 & min:0.2  \\
\hline
\end{tabular}      & \begin{tabular}{ll}
\hline
 $\mu$:8.31    & max:17.2 \\
 $\sigma$:6.08 & min:-1.4 \\
\hline
\end{tabular}      \\
 2018 & \begin{tabular}{ll}
\hline
 $\mu$:11.08   & max:20.0 \\
 $\sigma$:6.52 & min:-0.6 \\
\hline
\end{tabular}      & \begin{tabular}{ll}
\hline
 $\mu$:10.15   & max:20.3 \\
 $\sigma$:6.54 & min:0.2  \\
\hline
\end{tabular}      & \begin{tabular}{ll}
\hline
 $\mu$:8.7     & max:17.1 \\
 $\sigma$:5.03 & min:0.1  \\
\hline
\end{tabular}      & \begin{tabular}{ll}
\hline
 $\mu$:9.22    & max:19.9 \\
 $\sigma$:6.46 & min:-0.3 \\
\hline
\end{tabular}      \\
 2019 & \begin{tabular}{ll}
\hline
 $\mu$:11.38  & max:20.4 \\
 $\sigma$:5.3 & min:0.5  \\
\hline
\end{tabular}      & \begin{tabular}{ll}
\hline
 $\mu$:9.67    & max:20.4 \\
 $\sigma$:5.96 & min:0.3  \\
\hline
\end{tabular}      & \begin{tabular}{ll}
\hline
 $\mu$:9.12    & max:18.7 \\
 $\sigma$:4.75 & min:0.5  \\
\hline
\end{tabular}      & \begin{tabular}{ll}
\hline
 $\mu$:8.89    & max:19.4 \\
 $\sigma$:5.97 & min:-0.3 \\
\hline
\end{tabular}      \\
 2020 & \begin{tabular}{ll}
\hline
 $\mu$:12.26   & max:20.4 \\
 $\sigma$:4.77 & min:1.2  \\
\hline
\end{tabular}      & \begin{tabular}{ll}
\hline
 $\mu$:9.82    & max:20.5 \\
 $\sigma$:6.13 & min:-0.7 \\
\hline
\end{tabular}      & \begin{tabular}{ll}
\hline
 $\mu$:9.17   & max:17.9 \\
 $\sigma$:4.9 & min:0.5  \\
\hline
\end{tabular}      & \begin{tabular}{ll}
\hline
 $\mu$:9.35    & max:18.3 \\
 $\sigma$:6.15 & min:-0.4 \\
\hline
\end{tabular}      \\
 2021 & \begin{tabular}{ll}
\hline
 $\mu$:12.57   & max:24.2 \\
 $\sigma$:6.03 & min:0.0  \\
\hline
\end{tabular}      & \begin{tabular}{ll}
\hline
 $\mu$:10.01   & max:19.1 \\
 $\sigma$:5.99 & min:0.1  \\
\hline
\end{tabular}      & \begin{tabular}{ll}
\hline
 $\mu$:9.78    & max:17.8 \\
 $\sigma$:4.93 & min:0.6  \\
\hline
\end{tabular}      & \begin{tabular}{ll}
\hline
 $\mu$:nan    & max:nan \\
 $\sigma$:nan & min:nan \\
\hline
\end{tabular}      \\
 2022 & \begin{tabular}{ll}
\hline
 $\mu$:12.15   & max:22.4 \\
 $\sigma$:5.25 & min:0.9  \\
\hline
\end{tabular}      & \begin{tabular}{ll}
\hline
 $\mu$:12.99   & max:19.2 \\
 $\sigma$:3.61 & min:5.1  \\
\hline
\end{tabular}      & \begin{tabular}{ll}
\hline
 $\mu$:9.34    & max:16.1 \\
 $\sigma$:4.29 & min:0.7  \\
\hline
\end{tabular}      & \begin{tabular}{ll}
\hline
 $\mu$:9.09   & max:18.0 \\
 $\sigma$:6.2 & min:-1.1 \\
\hline
\end{tabular}      \\
\hline
\end{tabular}
	\caption[Table of station statistics for soil 20cm part 4]{The table shows the statistics of each station for each feature, except for Time as it is a strictly monotonic increasing sequence, for each year. The station names can be found in table \ref{tab:station:names}. $\mu$ Denotes the mean temperature, $\sigma$ denotes the standard deviation, "min" is the minimum temperature, and "max" is the maximum temperature. All values in the table have the unit degree Celsius.}
\end{table}
