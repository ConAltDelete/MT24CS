\section{Results}\todo{Summerize results}

\todo{Make sections}
\todo{Include tables}

\begin{table}
	\centering
	\begin{adjustwidth}{-1in}{-1in}
		\resizebox{1.35\textwidth}{!}{
			\begin{tabular}{llrrrrrrrrr}
\hline
 model                               &       &     52 &     37 &     50 &    38 &    57 &     34 &     27 &    17 &   average \\
\hline
 Linear model 10cm                   & $R^2$ &  0.221 &  0.473 &  0.584 & 0.388 & 0.295 & -1.386 &  0.551 & 0.501 &     0.423 \\
                                     & MAE   &  2.889 &  3.503 &  2.702 & 3.741 & 3.356 &  3.415 &  3.236 & 3.432 &     3.267 \\
                                     & RMSE  &  3.679 &  4.501 &  3.611 & 4.815 & 4.293 &  4.262 &  4.272 & 4.518 &     4.231 \\
                                     & bias  &  1.226 &  2.07  &  1.766 & 2.502 & 2.775 &  3.176 &  2.078 & 2.506 &     2.303 \\
 Linear model 20cm                   & $R^2$ &  0.604 &  0.391 &  0.434 & 0.308 & 0.125 & -2.248 &  0.415 & 0.336 &     0.308 \\
                                     & MAE   &  2.841 &  3.675 &  3.025 & 3.682 & 3.636 &  3.675 &  3.547 & 3.84  &     3.474 \\
                                     & RMSE  &  3.556 &  4.754 &  4.048 & 4.832 & 4.655 &  4.583 &  4.672 & 5.049 &     4.504 \\
                                     & bias  &  0.559 &  2.174 &  2.207 & 2.601 & 3.153 &  3.471 &  2.535 & 2.939 &     2.487 \\
 Plauborg model (daily values) 10cm  & $R^2$ &  0.664 &  0.873 &  0.938 & 0.845 & 0.85  &  0.397 &  0.895 & 0.875 &     0.861 \\
                                     & MAE   &  1.837 &  1.755 &  1.105 & 1.908 & 1.547 &  1.687 &  1.627 & 1.78  &     1.621 \\
                                     & RMSE  &  2.418 &  2.206 &  1.395 & 2.42  & 1.978 &  2.143 &  2.063 & 2.26  &     2.074 \\
                                     & bias  & -0.636 &  0.373 & -0.046 & 0.667 & 1.108 &  1.535 &  0.396 & 0.921 &     0.608 \\
 Plauborg model (daily values) 20cm  & $R^2$ &  0.83  &  0.905 &  0.946 & 0.912 & 0.863 &  0.316 &  0.901 & 0.844 &     0.876 \\
                                     & MAE   &  1.873 &  1.496 &  0.985 & 1.367 & 1.538 &  1.836 &  1.534 & 1.911 &     1.536 \\
                                     & RMSE  &  2.33  &  1.877 &  1.251 & 1.721 & 1.841 &  2.104 &  1.924 & 2.448 &     1.91  \\
                                     & bias  & -1.402 &  0.353 &  0.096 & 0.515 & 1.427 &  1.816 &  0.753 & 1.401 &     0.644 \\
 Plauborg model (hourly values) 10cm & $R^2$ &  0.636 &  0.857 &  0.884 & 0.754 & 0.677 & -0.193 &  0.872 & 0.828 &     0.794 \\
                                     & MAE   &  1.964 &  1.804 &  1.472 & 2.368 & 2.263 &  2.306 &  1.724 & 1.979 &     1.926 \\
                                     & RMSE  &  2.514 &  2.344 &  1.908 & 3.055 & 2.906 &  3.013 &  2.277 & 2.649 &     2.529 \\
                                     & bias  & -0.349 &  0.237 &  0.558 & 1.363 & 0.677 &  0.845 &  0.163 & 0.065 &     0.597 \\
 Plauborg model (hourly values) 20cm & $R^2$ &  0.803 &  0.83  &  0.836 & 0.736 & 0.617 & -0.547 &  0.839 & 0.762 &     0.756 \\
                                     & MAE   &  1.976 &  1.938 &  1.7   & 2.323 & 2.419 &  2.427 &  1.885 & 2.265 &     2.06  \\
                                     & RMSE  &  2.504 &  2.513 &  2.176 & 2.983 & 3.079 &  3.163 &  2.455 & 3.023 &     2.676 \\
                                     & bias  & -1.2   &  0.067 &  0.815 & 1.149 & 0.744 &  0.797 &  0.335 & 0.137 &     0.528 \\
 BiLSTM 20cm                         & $R^2$ &  0.776 &  0.873 &  0.869 & 0.844 & 0.795 &  0.195 &  0.862 & 0.871 &     0.825 \\
                                     & MAE   &  2.132 &  1.829 &  1.5   & 1.736 & 1.722 &  1.86  &  1.864 & 1.732 &     1.798 \\
                                     & RMSE  &  2.615 &  2.138 &  1.904 & 2.258 & 2.229 &  2.301 &  2.234 & 2.2   &     2.233 \\
                                     & bias  & -1.297 &  0.257 &  0.334 & 0.667 & 1.05  &  1.297 &  0.437 & 0.867 &     0.523 \\
 BiLSTM 10cm                         & $R^2$ &  0.699 &  0.929 &  0.948 & 0.919 & 0.898 &  0.64  &  0.924 & 0.927 &     0.906 \\
                                     & MAE   &  1.706 &  1.37  &  1.048 & 1.315 & 1.274 &  1.262 &  1.482 & 1.39  &     1.355 \\
                                     & RMSE  &  2.276 &  1.65  &  1.267 & 1.748 & 1.631 &  1.666 &  1.757 & 1.724 &     1.706 \\
                                     & bias  & -1.077 & -0.198 & -0.376 & 0.221 & 0.302 &  0.654 & -0.18  & 0.334 &     0.038 \\
 BiLSTM 20cm                         & $R^2$ &  0.799 &  0.934 &  0.962 & 0.946 & 0.922 &  0.654 &  0.942 & 0.921 &     0.918 \\
                                     & MAE   &  2.093 &  1.268 &  0.833 & 1.072 & 1.106 &  1.18  &  1.163 & 1.319 &     1.213 \\
                                     & RMSE  &  2.52  &  1.555 &  1.04  & 1.342 & 1.386 &  1.505 &  1.464 & 1.733 &     1.543 \\
                                     & bias  & -1.864 & -0.282 & -0.214 & 0.095 & 0.633 &  0.907 &  0.13  & 0.594 &     0.056 \\
 LSTM 10cm                           & $R^2$ &  0.703 &  0.917 &  0.936 & 0.891 & 0.859 &  0.508 &  0.91  & 0.917 &     0.884 \\
                                     & MAE   &  1.712 &  1.496 &  1.129 & 1.501 & 1.48  &  1.491 &  1.575 & 1.443 &     1.491 \\
                                     & RMSE  &  2.264 &  1.778 &  1.413 & 2.029 & 1.915 &  1.946 &  1.905 & 1.839 &     1.892 \\
                                     & bias  & -0.77  &  0.082 & -0.067 & 0.562 & 0.537 &  0.868 &  0.05  & 0.502 &     0.31  \\
 LSTM 20cm                           & $R^2$ &  0.803 &  0.904 &  0.932 & 0.901 & 0.848 &  0.432 &  0.903 & 0.893 &     0.874 \\
                                     & MAE   &  2.069 &  1.58  &  1.108 & 1.389 & 1.5   &  1.482 &  1.571 & 1.584 &     1.516 \\
                                     & RMSE  &  2.494 &  1.881 &  1.395 & 1.818 & 1.935 &  1.927 &  1.9   & 2.024 &     1.913 \\
                                     & bias  & -1.628 & -0.109 &  0.058 & 0.421 & 0.654 &  0.919 &  0.235 & 0.652 &     0.226 \\
 GRU 20cm                            & $R^2$ &  0.81  &  0.957 &  0.981 & 0.96  & 0.941 &  0.695 &  0.961 & 0.933 &     0.937 \\
                                     & MAE   &  2.029 &  1.016 &  0.573 & 0.883 & 0.948 &  1.112 &  0.898 & 1.088 &     1.026 \\
                                     & RMSE  &  2.435 &  1.25  &  0.725 & 1.147 & 1.204 &  1.432 &  1.194 & 1.592 &     1.349 \\
                                     & bias  & -1.907 & -0.351 & -0.207 & 0.087 & 0.638 &  0.913 &  0.106 & 0.566 &     0.048 \\
 GRU 10cm                            & $R^2$ &  0.723 &  0.931 &  0.946 & 0.903 & 0.86  &  0.497 &  0.923 & 0.926 &     0.894 \\
                                     & MAE   &  1.577 &  1.331 &  1.001 & 1.387 & 1.485 &  1.52  &  1.396 & 1.308 &     1.396 \\
                                     & RMSE  &  2.183 &  1.618 &  1.291 & 1.91  & 1.909 &  1.996 &  1.762 & 1.734 &     1.807 \\
                                     & bias  & -0.77  &  0.04  &  0.043 & 0.628 & 0.527 &  0.844 &  0.053 & 0.392 &     0.329 \\
\hline
\end{tabular}
		}
	\end{adjustwidth}
\end{table}


\begin{table}
	\centering
	\begin{adjustwidth}{-1in}{-1in}
		\resizebox{1.35\textwidth}{!}{
			\begin{tabular}{llrrrrrrrrr}
\hline
 model                               &       &     41 &   118 &     42 &     30 &    39 &     15 &    26 &     11 &   average \\
\hline
 Linear model 10cm                   & $R^2$ &  0.593 & 0.306 &  0.506 &  0.588 & 0.213 & -0.003 & 0.456 &  0.51  &     0.423 \\
                                     & MAE   &  3.07  & 3.486 &  3.525 &  3.106 & 3.193 &  3.141 & 3.651 &  2.713 &     3.267 \\
                                     & RMSE  &  3.976 & 4.5   &  4.571 &  4.053 & 4.057 &  3.918 & 4.714 &  3.567 &     4.231 \\
                                     & bias  &  1.494 & 2.93  &  2.109 &  1.733 & 2.835 &  2.799 & 2.902 &  1.529 &     2.303 \\
 Linear model 20cm                   & $R^2$ &  0.491 & 0.162 &  0.393 &  0.456 & 0.081 & -0.241 & 0.302 &  0.35  &     0.308 \\
                                     & MAE   &  3.286 & 3.654 &  3.741 &  3.433 & 3.39  &  3.342 & 4.009 &  2.924 &     3.474 \\
                                     & RMSE  &  4.248 & 4.726 &  4.863 &  4.465 & 4.31  &  4.198 & 5.17  &  3.821 &     4.504 \\
                                     & bias  &  1.677 & 3.208 &  2.364 &  2.015 & 3.083 &  3.006 & 3.282 &  1.692 &     2.487 \\
 Plauborg model (daily values) 10cm  & $R^2$ &  0.9   & 0.839 &  0.881 &  0.908 & 0.828 &  0.787 & 0.855 &  0.864 &     0.861 \\
                                     & MAE   &  1.587 & 1.697 &  1.75  &  1.519 & 1.455 &  1.428 & 1.937 &  1.504 &     1.621 \\
                                     & RMSE  &  1.975 & 2.165 &  2.239 &  1.914 & 1.896 &  1.806 & 2.43  &  1.879 &     2.074 \\
                                     & bias  & -0.293 & 1.137 &  0.333 & -0.046 & 1.193 &  1.114 & 1.251 &  0.339 &     0.608 \\
 Plauborg model (daily values) 20cm  & $R^2$ &  0.914 & 0.886 &  0.901 &  0.91  & 0.852 &  0.801 & 0.833 &  0.866 &     0.876 \\
                                     & MAE   &  1.409 & 1.384 &  1.54  &  1.471 & 1.468 &  1.411 & 2.101 &  1.443 &     1.536 \\
                                     & RMSE  &  1.748 & 1.742 &  1.966 &  1.817 & 1.729 &  1.681 & 2.528 &  1.735 &     1.91  \\
                                     & bias  & -0.371 & 1.139 &  0.346 & -0.014 & 1.402 &  1.215 & 1.578 &  0.463 &     0.644 \\
 Plauborg model (hourly values) 10cm & $R^2$ &  0.903 & 0.706 &  0.852 &  0.892 & 0.633 &  0.562 & 0.843 &  0.823 &     0.794 \\
                                     & MAE   &  1.519 & 2.322 &  1.885 &  1.555 & 2.151 &  1.991 & 1.918 &  1.666 &     1.926 \\
                                     & RMSE  &  1.938 & 2.928 &  2.501 &  2.072 & 2.77  &  2.589 & 2.532 &  2.146 &     2.529 \\
                                     & bias  &  0.151 & 1.639 &  0.703 &  0.354 & 0.701 &  0.903 & 0.821 &  0.038 &     0.597 \\
 Plauborg model (hourly values) 20cm & $R^2$ &  0.872 & 0.656 &  0.807 &  0.859 & 0.615 &  0.484 & 0.801 &  0.755 &     0.756 \\
                                     & MAE   &  1.665 & 2.422 &  2.099 &  1.708 & 2.186 &  2.094 & 2.105 &  1.847 &     2.06  \\
                                     & RMSE  &  2.135 & 3.029 &  2.739 &  2.276 & 2.79  &  2.706 & 2.757 &  2.346 &     2.676 \\
                                     & bias  &  0.13  & 1.722 &  0.746 &  0.428 & 0.633 &  0.827 & 0.892 & -0.023 &     0.528 \\
 BiLSTM 20cm                         & $R^2$ &  0.857 & 0.787 &  0.854 &  0.859 & 0.768 &  0.683 & 0.848 &  0.82  &     0.825 \\
                                     & MAE   &  1.862 & 1.829 &  1.935 &  1.887 & 1.642 &  1.617 & 1.932 &  1.627 &     1.798 \\
                                     & RMSE  &  2.202 & 2.351 &  2.349 &  2.234 & 2.128 &  2.07  & 2.385 &  1.992 &     2.233 \\
                                     & bias  & -0.21  & 1.355 &  0.407 &  0.045 & 0.976 &  0.979 & 1.258 &  0.172 &     0.523 \\
 BiLSTM 10cm                         & $R^2$ &  0.913 & 0.914 &  0.921 &  0.922 & 0.87  &  0.848 & 0.919 &  0.883 &     0.906 \\
                                     & MAE   &  1.525 & 1.224 &  1.464 &  1.439 & 1.252 &  1.188 & 1.474 &  1.437 &     1.355 \\
                                     & RMSE  &  1.828 & 1.575 &  1.827 &  1.759 & 1.643 &  1.518 & 1.81  &  1.743 &     1.706 \\
                                     & bias  & -0.676 & 0.723 & -0.108 & -0.484 & 0.366 &  0.378 & 0.669 & -0.257 &     0.038 \\
 BiLSTM 20cm                         & $R^2$ &  0.929 & 0.935 &  0.937 &  0.933 & 0.911 &  0.896 & 0.919 &  0.916 &     0.918 \\
                                     & MAE   &  1.291 & 1.07  &  1.254 &  1.269 & 1.055 &  0.953 & 1.408 &  1.111 &     1.213 \\
                                     & RMSE  &  1.581 & 1.315 &  1.562 &  1.559 & 1.337 &  1.211 & 1.753 &  1.369 &     1.543 \\
                                     & bias  & -0.767 & 0.735 & -0.112 & -0.483 & 0.603 &  0.512 & 0.875 & -0.145 &     0.056 \\
 LSTM 10cm                           & $R^2$ &  0.909 & 0.876 &  0.903 &  0.908 & 0.832 &  0.789 & 0.901 &  0.863 &     0.884 \\
                                     & MAE   &  1.557 & 1.446 &  1.604 &  1.544 & 1.412 &  1.403 & 1.617 &  1.523 &     1.491 \\
                                     & RMSE  &  1.872 & 1.895 &  2.016 &  1.906 & 1.866 &  1.788 & 2.002 &  1.885 &     1.892 \\
                                     & bias  & -0.364 & 1.059 &  0.214 & -0.153 & 0.588 &  0.635 & 0.896 & -0.048 &     0.31  \\
 LSTM 20cm                           & $R^2$ &  0.9   & 0.878 &  0.897 &  0.904 & 0.843 &  0.781 & 0.882 &  0.838 &     0.874 \\
                                     & MAE   &  1.553 & 1.368 &  1.595 &  1.538 & 1.337 &  1.369 & 1.699 &  1.547 &     1.516 \\
                                     & RMSE  &  1.874 & 1.795 &  1.996 &  1.872 & 1.772 &  1.757 & 2.12  &  1.907 &     1.913 \\
                                     & bias  & -0.487 & 1.029 &  0.174 & -0.17  & 0.57  &  0.582 & 1.009 & -0.066 &     0.226 \\
 GRU 20cm                            & $R^2$ &  0.954 & 0.951 &  0.952 &  0.957 & 0.927 &  0.915 & 0.937 &  0.944 &     0.937 \\
                                     & MAE   &  1.049 & 0.917 &  1.059 &  0.995 & 0.94  &  0.862 & 1.179 &  0.892 &     1.026 \\
                                     & RMSE  &  1.264 & 1.143 &  1.362 &  1.24  & 1.206 &  1.089 & 1.547 &  1.122 &     1.349 \\
                                     & bias  & -0.775 & 0.735 & -0.105 & -0.471 & 0.607 &  0.548 & 0.853 & -0.119 &     0.048 \\
 GRU 10cm                            & $R^2$ &  0.926 & 0.885 &  0.919 &  0.924 & 0.828 &  0.779 & 0.913 &  0.872 &     0.894 \\
                                     & MAE   &  1.381 & 1.411 &  1.41  &  1.367 & 1.443 &  1.452 & 1.487 &  1.431 &     1.396 \\
                                     & RMSE  &  1.691 & 1.844 &  1.846 &  1.73  & 1.887 &  1.831 & 1.883 &  1.82  &     1.807 \\
                                     & bias  & -0.295 & 1.124 &  0.266 & -0.102 & 0.587 &  0.665 & 0.853 & -0.005 &     0.329 \\
\hline
\end{tabular}
		}
	\end{adjustwidth}
\end{table}

\subsection{Linear regression vs Plauborg}

The global measure for the linear regression has an average error of $2.3C^\circ \pm 4.23 C^\circ$ while the global messure of the Plauborg daily model has an average error of $0.6C^\circ \pm 1.96 C^\circ$. Further more Plauborg has an hight $R^2$ value indicating that it follows the temperature changes in the soil better than just scaling the air temperature by a scaling factor.

\subsection{Modification of Plauborg}

The Plauborg model trained in Norway was found to only need 3 days ($t_0,t_{-1},t_{-2}$) compared to \cite{plauborg_simple_2002} that needed 4 days ($t_0,t_{-1},t_{-2},t_{-3}$). However for the Fourier terms both models (Danish model and the Norwegian model) required 2 sine and cosine terms. For the 20cm target the models diverge in the sense of quantity of terms. It was found that the 20cm model needs 14 sine terms and 2 cosine terms, however only needs 2 days.
\begin{figure}
	\begin{subfigure}{0.45\textwidth}
		\centering
		\includegraphics[width=0.7\linewidth]{../../results/plots/diffplot_Plauborg_day_stat_10_Innlandet_2022_TJM10}
		\caption[Plauborg daily TJM10]{The daily model of Plauborg model}
		\label{fig:diffplotplauborgdaystat10innlandet2022tjm10}
	\end{subfigure}
	\begin{subfigure}{0.45\textwidth}
		\centering
		\includegraphics[width=0.7\linewidth]{../../results/plots/diffplot_Plauborg_stat_10_Innlandet_2022_TJM10}
		\caption[Plauborg hourly TJM10]{The hourly model of Plauborg model}
		\label{fig:diffplotplauborgstat10innlandet2022tjm10}
	\end{subfigure}
	\caption{Comperasion of daily versus hourly predictions}
\end{figure}

The modification to Plauborg's model is minor, by replacing the $\omega$ with a larger coefficient it can be used with hourly data. As seen in figure \ref{fig:diffplotplauborgstat10innlandet2022tjm10} the variation is stronger than \ref{fig:diffplotplauborgdaystat10innlandet2022tjm10} however the overall performance is comparable as seen in table \ref{tab:Plauborg:day:10} and table \ref{tab:Plauborg:hour:10}. 

\begin{table}
	\centering
	\resizebox{\textwidth}{!}{
		\begin{tabular}{llrrrrrr}
\hline
 scope   & spesific
scope           &       RMSE
[℃] &   MAE [℃] &        bias
[℃] &   log($\kappa$(model)) &    digit
sensitivity &     R² \\
\hline
 global  & ---       & 2.529 &     1.926 &  0.597 &                 -0.434 & -1 &  0.794 \\
 region  & Østfold   & 2.448 &     1.894 &  0.512 &                 -0.434 & -1 &  0.816 \\
 region  & Vestfold  & 2.412 &     1.81  &  0.733 &                 -0.444 & -1 &  0.846 \\
 region  & Trøndelag & 2.822 &     2.176 &  0.781 &                 -0.442 & -1 &  0.547 \\
 region  & Innlandet & 2.382 &     1.805 &  0.312 &                 -0.449 & -1 &  0.847 \\
 local   & 52        & 2.514 &     1.964 & -0.349 &                 -0.445 & -1 &  0.636 \\
 local   & 41        & 1.938 &     1.519 &  0.151 &                 -0.444 & -1 &  0.903 \\
 local   & 37        & 2.344 &     1.804 &  0.237 &                 -0.439 & -1 &  0.857 \\
 local   & 118       & 2.928 &     2.322 &  1.639 &                 -0.442 & -1 &  0.706 \\
 local   & 50        & 1.908 &     1.472 &  0.558 &                 -0.448 & -1 &  0.884 \\
 local   & 42        & 2.501 &     1.885 &  0.703 &                 -0.44  & -1 &  0.852 \\
 local   & 38        & 3.055 &     2.368 &  1.363 &                 -0.442 & -1 &  0.754 \\
 local   & 30        & 2.072 &     1.555 &  0.354 &                 -0.436 & -1 &  0.892 \\
 local   & 57        & 2.906 &     2.263 &  0.677 &                 -0.448 & -1 &  0.677 \\
 local   & 39        & 2.77  &     2.151 &  0.701 &                 -0.438 & -1 &  0.633 \\
 local   & 34        & 3.013 &     2.306 &  0.845 &                 -0.444 & -1 & -0.193 \\
 local   & 15        & 2.589 &     1.991 &  0.903 &                 -0.443 & -1 &  0.562 \\
 local   & 27        & 2.277 &     1.724 &  0.163 &                 -0.444 & -1 &  0.872 \\
 local   & 26        & 2.532 &     1.918 &  0.821 &                 -0.44  & -1 &  0.843 \\
 local   & 17        & 2.649 &     1.979 &  0.065 &                 -0.45  & -1 &  0.828 \\
 local   & 11        & 2.146 &     1.666 &  0.038 &                 -0.445 & -1 &  0.823 \\
\hline
\end{tabular}
	}
	\caption{Hourly Plauborg model results.}
	\label{tab:plauborg_hour_res}
\end{table}

\begin{table}
	\centering
	\resizebox{\textwidth}{!}{
		\begin{tabular}{llrrrrrr}
\hline
 scope   & spesific
scope           &       RMSE
[℃] &   MAE [℃] &        bias
[℃] &   log($\kappa$(model)) &    digit
sensitivity &    R² \\
\hline
 global  & ---       & 2.074 &     1.621 &  0.608 &                 -1.271 & -2 & 0.861 \\
 region  & Østfold   & 2.168 &     1.704 &  0.24  &                 -1.256 & -2 & 0.856 \\
 region  & Vestfold  & 2.022 &     1.564 &  0.219 &                 -1.265 & -2 & 0.892 \\
 region  & Trøndelag & 1.957 &     1.528 &  1.235 &                 -1.266 & -2 & 0.782 \\
 region  & Innlandet & 2.165 &     1.71  &  0.714 &                 -1.265 & -2 & 0.873 \\
 local   & 52        & 2.418 &     1.837 & -0.636 &                 -1.257 & -2 & 0.664 \\
 local   & 41        & 1.975 &     1.587 & -0.293 &                 -1.273 & -2 & 0.9   \\
 local   & 37        & 2.206 &     1.755 &  0.373 &                 -1.265 & -2 & 0.873 \\
 local   & 118       & 2.165 &     1.697 &  1.137 &                 -1.266 & -2 & 0.839 \\
 local   & 50        & 1.395 &     1.105 & -0.046 &                 -1.261 & -2 & 0.938 \\
 local   & 42        & 2.239 &     1.75  &  0.333 &                 -1.263 & -2 & 0.881 \\
 local   & 38        & 2.42  &     1.908 &  0.667 &                 -1.264 & -2 & 0.845 \\
 local   & 30        & 1.914 &     1.519 & -0.046 &                 -1.264 & -2 & 0.908 \\
 local   & 57        & 1.978 &     1.547 &  1.108 &                 -1.266 & -2 & 0.85  \\
 local   & 39        & 1.896 &     1.455 &  1.193 &                 -1.269 & -2 & 0.828 \\
 local   & 34        & 2.143 &     1.687 &  1.535 &                 -1.271 & -2 & 0.397 \\
 local   & 15        & 1.806 &     1.428 &  1.114 &                 -1.267 & -2 & 0.787 \\
 local   & 27        & 2.063 &     1.627 &  0.396 &                 -1.256 & -2 & 0.895 \\
 local   & 26        & 2.43  &     1.937 &  1.251 &                 -1.269 & -2 & 0.855 \\
 local   & 17        & 2.26  &     1.78  &  0.921 &                 -1.264 & -2 & 0.875 \\
 local   & 11        & 1.879 &     1.504 &  0.339 &                 -1.261 & -2 & 0.864 \\
\hline
\end{tabular}
	}
	\caption{Daily Plauborg model results.}
	\label{tab:plauborg_day_res}
\end{table}

With modification to the model to accept hourly data it still preforms approximately as well as the daily data version. With a average error of $0.597C^\circ \pm 2.529C^\circ$ for TJM10 and $0.528C^\circ \pm 2.676C^\circ$ for TJM20. It was found that the modified Plauborg model only needs 2 sine terms to make a good prediction and 12h of air temperature which would translate to half a day instead of 3 days.

\subsection{Deep learning models}



\subsection{Model comperison}

The Plauborg has a clear advantage over linear regression.
