\section{Results}\todo{Summerize results}

\todo{Make sections}
\todo{Include tables}

\begin{table}
	\centering
	\begin{adjustwidth}{-1in}{-1in}
		\resizebox{1.35\textwidth}{!}{
			\begin{tabular}{llrrrrrrrrr}
\hline
 model                               &       &    52 &    37 &    50 &   38 &   57 &    34 &    27 &   17 &   average \\
\hline
 Linear model 10cm                   & $R^2$ &  0.22 &  0.47 &  0.58 & 0.39 & 0.3  & -1.39 &  0.55 & 0.5  &      0.42 \\
                                     & MAE   &  2.89 &  3.5  &  2.7  & 3.74 & 3.36 &  3.42 &  3.24 & 3.43 &      3.27 \\
                                     & RMSE  &  3.68 &  4.5  &  3.61 & 4.81 & 4.29 &  4.26 &  4.27 & 4.52 &      4.23 \\
                                     & bias  &  1.23 &  2.07 &  1.77 & 2.5  & 2.77 &  3.18 &  2.08 & 2.51 &      2.3  \\
 Plauborg model (daily values) 10cm  & $R^2$ &  0.66 &  0.87 &  0.94 & 0.85 & 0.85 &  0.4  &  0.9  & 0.88 &      0.86 \\
                                     & MAE   &  1.84 &  1.76 &  1.11 & 1.91 & 1.55 &  1.69 &  1.63 & 1.78 &      1.62 \\
                                     & RMSE  &  2.42 &  2.21 &  1.4  & 2.42 & 1.98 &  2.14 &  2.06 & 2.26 &      2.07 \\
                                     & bias  & -0.64 &  0.37 & -0.05 & 0.67 & 1.11 &  1.53 &  0.4  & 0.92 &      0.61 \\
 Plauborg model (hourly values) 10cm & $R^2$ &  0.64 &  0.86 &  0.88 & 0.75 & 0.68 & -0.19 &  0.87 & 0.83 &      0.79 \\
                                     & MAE   &  1.96 &  1.8  &  1.47 & 2.37 & 2.26 &  2.31 &  1.72 & 1.98 &      1.93 \\
                                     & RMSE  &  2.51 &  2.34 &  1.91 & 3.05 & 2.91 &  3.01 &  2.28 & 2.65 &      2.53 \\
                                     & bias  & -0.35 &  0.24 &  0.56 & 1.36 & 0.68 &  0.85 &  0.16 & 0.06 &      0.6  \\
 BiLSTM 10cm                         & $R^2$ &  0.75 &  0.96 &  0.96 & 0.95 & 0.92 &  0.69 &  0.95 & 0.95 &      0.93 \\
                                     & MAE   &  1.48 &  1.01 &  0.86 & 1.03 & 1.11 &  1.17 &  1.13 & 1.11 &      1.11 \\
                                     & RMSE  &  2.08 &  1.22 &  1.06 & 1.43 & 1.4  &  1.56 &  1.36 & 1.4  &      1.42 \\
                                     & bias  & -1.08 & -0.17 & -0.48 & 0.18 & 0.42 &  0.75 & -0.11 & 0.46 &      0.06 \\
 LSTM 10cm                           & $R^2$ &  0.71 &  0.92 &  0.94 & 0.89 & 0.86 &  0.52 &  0.91 & 0.92 &      0.89 \\
                                     & MAE   &  1.72 &  1.49 &  1.11 & 1.48 & 1.46 &  1.47 &  1.56 & 1.43 &      1.47 \\
                                     & RMSE  &  2.25 &  1.77 &  1.39 & 2.01 & 1.91 &  1.92 &  1.89 & 1.81 &      1.87 \\
                                     & bias  & -0.81 &  0.02 & -0.05 & 0.54 & 0.55 &  0.88 &  0.04 & 0.46 &      0.3  \\
 GRU 10cm                            & $R^2$ &  0.7  &  0.91 &  0.93 & 0.89 & 0.91 &  0.64 &  0.91 & 0.9  &      0.89 \\
                                     & MAE   &  1.69 &  1.53 &  1.18 & 1.57 & 1.19 &  1.31 &  1.49 & 1.54 &      1.42 \\
                                     & RMSE  &  2.29 &  1.89 &  1.44 & 2    & 1.5  &  1.69 &  1.85 & 1.98 &      1.81 \\
                                     & bias  & -1.19 & -0.15 & -0.72 & 0.02 & 0.59 &  1.03 & -0.14 & 0.5  &      0.03 \\
 BiGRU 10cm                          & $R^2$ &  0.69 &  0.93 &  0.95 & 0.92 & 0.88 &  0.62 &  0.92 & 0.93 &      0.9  \\
                                     & MAE   &  1.69 &  1.37 &  1.05 & 1.34 & 1.38 &  1.34 &  1.44 & 1.31 &      1.36 \\
                                     & RMSE  &  2.31 &  1.66 &  1.29 & 1.78 & 1.74 &  1.73 &  1.77 & 1.69 &      1.72 \\
                                     & bias  & -1.15 & -0.31 & -0.4  & 0.23 & 0.23 &  0.55 & -0.3  & 0.08 &     -0.04 \\
\hline
\end{tabular}
		}
	\end{adjustwidth}
\end{table}


\begin{table}
	\centering
	\begin{adjustwidth}{-1in}{-1in}
		\resizebox{1.35\textwidth}{!}{
			\begin{tabular}{llrrrrrrrrr}
\hline
 model                               &       &    41 &   118 &    42 &    30 &   39 &    15 &   26 &    11 &   average \\
\hline
 Linear model 10cm                   & $R^2$ &  0.59 &  0.31 &  0.51 &  0.59 & 0.21 & -0    & 0.46 &  0.51 &      0.42 \\
                                     & MAE   &  3.07 &  3.49 &  3.52 &  3.11 & 3.19 &  3.14 & 3.65 &  2.71 &      3.27 \\
                                     & RMSE  &  3.98 &  4.5  &  4.57 &  4.05 & 4.06 &  3.92 & 4.71 &  3.57 &      4.23 \\
                                     & bias  &  1.49 &  2.93 &  2.11 &  1.73 & 2.83 &  2.8  & 2.9  &  1.53 &      2.3  \\
 Plauborg model (daily values) 10cm  & $R^2$ &  0.9  &  0.84 &  0.88 &  0.91 & 0.83 &  0.79 & 0.86 &  0.86 &      0.86 \\
                                     & MAE   &  1.59 &  1.7  &  1.75 &  1.52 & 1.46 &  1.43 & 1.94 &  1.5  &      1.62 \\
                                     & RMSE  &  1.98 &  2.17 &  2.24 &  1.91 & 1.9  &  1.81 & 2.43 &  1.88 &      2.07 \\
                                     & bias  & -0.29 &  1.14 &  0.33 & -0.05 & 1.19 &  1.11 & 1.25 &  0.34 &      0.61 \\
 Plauborg model (hourly values) 10cm & $R^2$ &  0.9  &  0.71 &  0.85 &  0.89 & 0.63 &  0.56 & 0.84 &  0.82 &      0.79 \\
                                     & MAE   &  1.52 &  2.32 &  1.88 &  1.55 & 2.15 &  1.99 & 1.92 &  1.67 &      1.93 \\
                                     & RMSE  &  1.94 &  2.93 &  2.5  &  2.07 & 2.77 &  2.59 & 2.53 &  2.15 &      2.53 \\
                                     & bias  &  0.15 &  1.64 &  0.7  &  0.35 & 0.7  &  0.9  & 0.82 &  0.04 &      0.6  \\
 BiLSTM 10cm                         & $R^2$ &  0.94 &  0.94 &  0.95 &  0.95 & 0.88 &  0.89 & 0.95 &  0.93 &      0.93 \\
                                     & MAE   &  1.2  &  1.07 &  1.1  &  1.13 & 1.24 &  1.02 & 1.2  &  1.08 &      1.11 \\
                                     & RMSE  &  1.47 &  1.32 &  1.4  &  1.44 & 1.59 &  1.3  & 1.49 &  1.34 &      1.42 \\
                                     & bias  & -0.76 &  0.64 & -0.17 & -0.56 & 0.5  &  0.44 & 0.74 & -0.12 &      0.06 \\
 LSTM 10cm                           & $R^2$ &  0.91 &  0.88 &  0.9  &  0.91 & 0.83 &  0.8  & 0.91 &  0.87 &      0.89 \\
                                     & MAE   &  1.54 &  1.41 &  1.59 &  1.52 & 1.4  &  1.37 & 1.58 &  1.5  &      1.47 \\
                                     & RMSE  &  1.85 &  1.87 &  2    &  1.88 & 1.85 &  1.75 & 1.96 &  1.85 &      1.87 \\
                                     & bias  & -0.37 &  1.05 &  0.2  & -0.18 & 0.62 &  0.67 & 0.87 & -0.05 &      0.3  \\
 GRU 10cm                            & $R^2$ &  0.9  &  0.91 &  0.9  &  0.91 & 0.88 &  0.87 & 0.89 &  0.89 &      0.89 \\
                                     & MAE   &  1.6  &  1.29 &  1.64 &  1.51 & 1.21 &  1.09 & 1.65 &  1.37 &      1.42 \\
                                     & RMSE  &  1.98 &  1.62 &  2.06 &  1.88 & 1.55 &  1.4  & 2.08 &  1.72 &      1.81 \\
                                     & bias  & -0.94 &  0.44 & -0.31 & -0.7  & 0.67 &  0.53 & 0.77 & -0.19 &      0.03 \\
 BiGRU 10cm                          & $R^2$ &  0.91 &  0.92 &  0.92 &  0.92 & 0.87 &  0.84 & 0.93 &  0.88 &      0.9  \\
                                     & MAE   &  1.5  &  1.2  &  1.46 &  1.44 & 1.29 &  1.26 & 1.39 &  1.47 &      1.36 \\
                                     & RMSE  &  1.81 &  1.55 &  1.82 &  1.76 & 1.66 &  1.56 & 1.74 &  1.8  &      1.72 \\
                                     & bias  & -0.72 &  0.71 & -0.14 & -0.51 & 0.28 &  0.33 & 0.51 & -0.37 &     -0.04 \\
\hline
\end{tabular}
		}
	\end{adjustwidth}
\end{table}

\begin{table}
	\centering
	\begin{adjustwidth}{-1in}{-1in}
		\resizebox{1.35\textwidth}{!}{
			\begin{tabular}{llrrrrrrrrr}
\hline
 model                               &       &    52 &    37 &    50 &   38 &   57 &    34 &   27 &   17 &   average \\
\hline
 Linear model 20cm                   & $R^2$ &  0.6  &  0.39 &  0.43 & 0.31 & 0.13 & -2.25 & 0.42 & 0.34 &      0.31 \\
                                     & MAE   &  2.84 &  3.68 &  3.03 & 3.68 & 3.64 &  3.67 & 3.55 & 3.84 &      3.47 \\
                                     & RMSE  &  3.56 &  4.75 &  4.05 & 4.83 & 4.65 &  4.58 & 4.67 & 5.05 &      4.5  \\
                                     & bias  &  0.56 &  2.17 &  2.21 & 2.6  & 3.15 &  3.47 & 2.53 & 2.94 &      2.49 \\
 Plauborg model (daily values) 20cm  & $R^2$ &  0.83 &  0.91 &  0.95 & 0.91 & 0.86 &  0.32 & 0.9  & 0.84 &      0.88 \\
                                     & MAE   &  1.87 &  1.5  &  0.98 & 1.37 & 1.54 &  1.84 & 1.53 & 1.91 &      1.54 \\
                                     & RMSE  &  2.33 &  1.88 &  1.25 & 1.72 & 1.84 &  2.1  & 1.92 & 2.45 &      1.91 \\
                                     & bias  & -1.4  &  0.35 &  0.1  & 0.52 & 1.43 &  1.82 & 0.75 & 1.4  &      0.64 \\
 Plauborg model (hourly values) 20cm & $R^2$ &  0.8  &  0.83 &  0.84 & 0.74 & 0.62 & -0.55 & 0.84 & 0.76 &      0.76 \\
                                     & MAE   &  1.98 &  1.94 &  1.7  & 2.32 & 2.42 &  2.43 & 1.89 & 2.27 &      2.06 \\
                                     & RMSE  &  2.5  &  2.51 &  2.18 & 2.98 & 3.08 &  3.16 & 2.46 & 3.02 &      2.68 \\
                                     & bias  & -1.2  &  0.07 &  0.82 & 1.15 & 0.74 &  0.8  & 0.34 & 0.14 &      0.53 \\
 BiLSTM 20cm                         & $R^2$ &  0.8  &  0.92 &  0.95 & 0.93 & 0.9  &  0.59 & 0.93 & 0.91 &      0.9  \\
                                     & MAE   &  2.07 &  1.44 &  0.99 & 1.24 & 1.23 &  1.23 & 1.38 & 1.51 &      1.35 \\
                                     & RMSE  &  2.49 &  1.72 &  1.2  & 1.54 & 1.6  &  1.64 & 1.66 & 1.88 &      1.7  \\
                                     & bias  & -1.79 & -0.25 & -0.13 & 0.19 & 0.53 &  0.81 & 0.13 & 0.59 &      0.07 \\
 LSTM 20cm                           & $R^2$ &  0.83 &  0.94 &  0.96 & 0.95 & 0.87 &  0.41 & 0.92 & 0.85 &      0.89 \\
                                     & MAE   &  1.99 &  1.28 &  0.88 & 1.08 & 1.33 &  1.54 & 1.36 & 1.8  &      1.36 \\
                                     & RMSE  &  2.35 &  1.55 &  1.09 & 1.33 & 1.77 &  1.96 & 1.77 & 2.37 &      1.76 \\
                                     & bias  & -1.61 &  0.13 & -0.17 & 0.31 & 1.11 &  1.48 & 0.6  & 1.3  &      0.42 \\
 GRU 20cm                            & $R^2$ &  0.8  &  0.95 &  0.96 & 0.96 & 0.9  &  0.5  & 0.93 & 0.86 &      0.9  \\
                                     & MAE   &  2.15 &  1.15 &  0.85 & 0.99 & 1.19 &  1.49 & 1.2  & 1.63 &      1.29 \\
                                     & RMSE  &  2.52 &  1.41 &  1.06 & 1.22 & 1.57 &  1.83 & 1.63 & 2.33 &      1.7  \\
                                     & bias  & -1.81 & -0.04 & -0.35 & 0.14 & 0.98 &  1.34 & 0.36 & 1.11 &      0.24 \\
 BiGRU 20cm                          & $R^2$ &  0.79 &  0.93 &  0.96 & 0.94 & 0.91 &  0.63 & 0.94 & 0.93 &      0.92 \\
                                     & MAE   &  2.13 &  1.28 &  0.83 & 1.09 & 1.2  &  1.23 & 1.21 & 1.21 &      1.24 \\
                                     & RMSE  &  2.57 &  1.56 &  1.05 & 1.38 & 1.53 &  1.58 & 1.52 & 1.63 &      1.57 \\
                                     & bias  & -1.91 & -0.4  & -0.15 & 0.11 & 0.55 &  0.81 & 0.04 & 0.43 &      0.01 \\
\hline
\end{tabular}
		}
	\end{adjustwidth}
\end{table}


\begin{table}
	\centering
	\begin{adjustwidth}{-1in}{-1in}
		\resizebox{1.35\textwidth}{!}{
			\begin{tabular}{llrrrrrrrrr}
\hline
 model                               &       &    41 &   118 &    42 &    30 &   39 &    15 &   26 &    11 &   average \\
\hline
 Linear model 20cm                   & $R^2$ &  0.49 &  0.16 &  0.39 &  0.46 & 0.08 & -0.24 & 0.3  &  0.35 &      0.31 \\
                                     & MAE   &  3.29 &  3.65 &  3.74 &  3.43 & 3.39 &  3.34 & 4.01 &  2.92 &      3.47 \\
                                     & RMSE  &  4.25 &  4.73 &  4.86 &  4.46 & 4.31 &  4.2  & 5.17 &  3.82 &      4.5  \\
                                     & bias  &  1.68 &  3.21 &  2.36 &  2.02 & 3.08 &  3.01 & 3.28 &  1.69 &      2.49 \\
 Plauborg model (daily values) 20cm  & $R^2$ &  0.91 &  0.89 &  0.9  &  0.91 & 0.85 &  0.8  & 0.83 &  0.87 &      0.88 \\
                                     & MAE   &  1.41 &  1.38 &  1.54 &  1.47 & 1.47 &  1.41 & 2.1  &  1.44 &      1.54 \\
                                     & RMSE  &  1.75 &  1.74 &  1.97 &  1.82 & 1.73 &  1.68 & 2.53 &  1.74 &      1.91 \\
                                     & bias  & -0.37 &  1.14 &  0.35 & -0.01 & 1.4  &  1.22 & 1.58 &  0.46 &      0.64 \\
 Plauborg model (hourly values) 20cm & $R^2$ &  0.87 &  0.66 &  0.81 &  0.86 & 0.61 &  0.48 & 0.8  &  0.76 &      0.76 \\
                                     & MAE   &  1.67 &  2.42 &  2.1  &  1.71 & 2.19 &  2.09 & 2.11 &  1.85 &      2.06 \\
                                     & RMSE  &  2.14 &  3.03 &  2.74 &  2.28 & 2.79 &  2.71 & 2.76 &  2.35 &      2.68 \\
                                     & bias  &  0.13 &  1.72 &  0.75 &  0.43 & 0.63 &  0.83 & 0.89 & -0.02 &      0.53 \\
 BiLSTM 20cm                         & $R^2$ &  0.91 &  0.92 &  0.92 &  0.92 & 0.89 &  0.86 & 0.9  &  0.88 &      0.9  \\
                                     & MAE   &  1.45 &  1.16 &  1.46 &  1.38 & 1.12 &  1.08 & 1.57 &  1.33 &      1.35 \\
                                     & RMSE  &  1.74 &  1.5  &  1.78 &  1.67 & 1.5  &  1.4  & 1.91 &  1.61 &      1.7  \\
                                     & bias  & -0.68 &  0.82 & -0.03 & -0.4  & 0.46 &  0.43 & 0.89 & -0.21 &      0.07 \\
 LSTM 20cm                           & $R^2$ &  0.93 &  0.92 &  0.93 &  0.93 & 0.86 &  0.81 & 0.86 &  0.84 &      0.89 \\
                                     & MAE   &  1.34 &  1.12 &  1.34 &  1.31 & 1.25 &  1.2  & 1.8  &  1.42 &      1.36 \\
                                     & RMSE  &  1.61 &  1.48 &  1.64 &  1.61 & 1.68 &  1.62 & 2.32 &  1.9  &      1.76 \\
                                     & bias  & -0.58 &  0.92 &  0.16 & -0.18 & 1.01 &  0.89 & 1.49 &  0.43 &      0.42 \\
 GRU 20cm                            & $R^2$ &  0.92 &  0.94 &  0.93 &  0.92 & 0.88 &  0.84 & 0.87 &  0.84 &      0.9  \\
                                     & MAE   &  1.36 &  0.96 &  1.31 &  1.33 & 1.22 &  1.14 & 1.59 &  1.34 &      1.29 \\
                                     & RMSE  &  1.64 &  1.26 &  1.63 &  1.66 & 1.56 &  1.52 & 2.19 &  1.87 &      1.7  \\
                                     & bias  & -0.78 &  0.73 & -0.03 & -0.37 & 0.92 &  0.76 & 1.25 &  0.24 &      0.24 \\
 BiGRU 20cm                          & $R^2$ &  0.93 &  0.93 &  0.93 &  0.93 & 0.9  &  0.88 & 0.93 &  0.9  &      0.92 \\
                                     & MAE   &  1.32 &  1.06 &  1.26 &  1.28 & 1.1  &  1.05 & 1.31 &  1.2  &      1.24 \\
                                     & RMSE  &  1.57 &  1.35 &  1.59 &  1.55 & 1.4  &  1.31 & 1.68 &  1.47 &      1.57 \\
                                     & bias  & -0.75 &  0.76 & -0.1  & -0.46 & 0.51 &  0.49 & 0.76 & -0.22 &      0.01 \\
\hline
\end{tabular}
		}
	\end{adjustwidth}
\end{table}

\subsection{Linear regression vs Plauborg}

The global measure for the linear regression has an average error of $2.3^\circ C \pm 4.23^\circ C$ while the global messure of the Plauborg daily model has an average error of $0.6^\circ C \pm 1.96^\circ C$. Further more Plauborg has an hight $R^2$ value indicating that it follows the temperature changes in the soil better than just scaling the air temperature by a scaling factor.

The linear model shows subpar predictive capabilities compared to Plauborg's model who uses the same technics but interoperate time dependence. 

\subsection{Modification of Plauborg}

The Plauborg model trained in Norway was found to only need 3 days ($t_0,t_{-1},t_{-2}$) compared to \cite{plauborg_simple_2002} that needed 4 days ($t_0,t_{-1},t_{-2},t_{-3}$). However for the Fourier terms both models (Danish model and the Norwegian model) required 2 sine and cosine terms. For the 20cm target the models diverge in the sense of quantity of terms. It was found that the 20cm model needs 14 sine terms and 2 cosine terms, however only needs 2 days.
\begin{figure}[H]
	\begin{subfigure}{\textwidth}
		\centering
		\includegraphics[width=0.8\linewidth]{../../results/plots/diffplot_Plauborg_day_stat_10_Innlandet_2022_TJM10}
		\caption[Plauborg daily TJM10]{The daily model of Plauborg model. The model uses daily avergae tempratures to predict soil temperatures.}
		\label{fig:diffplotplauborgdaystat10innlandet2022tjm10}
	\end{subfigure}
	\begin{subfigure}{\textwidth}
		\centering
		\includegraphics[width=0.8\linewidth]{../../results/plots/diffplot_Plauborg_stat_10_Innlandet_2022_TJM10}
		\caption[Plauborg hourly TJM10]{The hourly model of Plauborg model. The model uses hourly temperature data.}
		\label{fig:diffplotplauborgstat10innlandet2022tjm10}
	\end{subfigure}
	\caption{Comperasion of daily versus hourly predictions}
\end{figure}

The modification to Plauborg's model is minor, by replacing the $\omega$ with a larger coefficient it can be used with hourly data. As seen in figure \ref{fig:diffplotplauborgstat10innlandet2022tjm10} the variation is stronger than \ref{fig:diffplotplauborgdaystat10innlandet2022tjm10} however the overall performance is comparable as seen in table \ref{tab:Plauborg:day:10} and table \ref{tab:Plauborg:hour:10}. 

\begin{table}[t]
	\centering
	\resizebox{\textwidth}{!}{
		\begin{tabular}{llrrrrrr}
\hline
 scope   & spesific
scope           &       RMSE
[℃] &   MAE [℃] &        bias
[℃] &   log($\kappa$(model)) &    digit
sensitivity &     R² \\
\hline
 global  & ---       & 2.529 &     1.926 &  0.597 &                 -0.434 & -1 &  0.794 \\
 region  & Østfold   & 2.448 &     1.894 &  0.512 &                 -0.434 & -1 &  0.816 \\
 region  & Vestfold  & 2.412 &     1.81  &  0.733 &                 -0.444 & -1 &  0.846 \\
 region  & Trøndelag & 2.822 &     2.176 &  0.781 &                 -0.442 & -1 &  0.547 \\
 region  & Innlandet & 2.382 &     1.805 &  0.312 &                 -0.449 & -1 &  0.847 \\
 local   & 52        & 2.514 &     1.964 & -0.349 &                 -0.445 & -1 &  0.636 \\
 local   & 41        & 1.938 &     1.519 &  0.151 &                 -0.444 & -1 &  0.903 \\
 local   & 37        & 2.344 &     1.804 &  0.237 &                 -0.439 & -1 &  0.857 \\
 local   & 118       & 2.928 &     2.322 &  1.639 &                 -0.442 & -1 &  0.706 \\
 local   & 50        & 1.908 &     1.472 &  0.558 &                 -0.448 & -1 &  0.884 \\
 local   & 42        & 2.501 &     1.885 &  0.703 &                 -0.44  & -1 &  0.852 \\
 local   & 38        & 3.055 &     2.368 &  1.363 &                 -0.442 & -1 &  0.754 \\
 local   & 30        & 2.072 &     1.555 &  0.354 &                 -0.436 & -1 &  0.892 \\
 local   & 57        & 2.906 &     2.263 &  0.677 &                 -0.448 & -1 &  0.677 \\
 local   & 39        & 2.77  &     2.151 &  0.701 &                 -0.438 & -1 &  0.633 \\
 local   & 34        & 3.013 &     2.306 &  0.845 &                 -0.444 & -1 & -0.193 \\
 local   & 15        & 2.589 &     1.991 &  0.903 &                 -0.443 & -1 &  0.562 \\
 local   & 27        & 2.277 &     1.724 &  0.163 &                 -0.444 & -1 &  0.872 \\
 local   & 26        & 2.532 &     1.918 &  0.821 &                 -0.44  & -1 &  0.843 \\
 local   & 17        & 2.649 &     1.979 &  0.065 &                 -0.45  & -1 &  0.828 \\
 local   & 11        & 2.146 &     1.666 &  0.038 &                 -0.445 & -1 &  0.823 \\
\hline
\end{tabular}
	}
	\caption{Hourly Plauborg model results.}
	\label{tab:plauborg_hour_res}
\end{table}

\begin{table}[t]
	\centering
	\resizebox{\textwidth}{!}{
		\begin{tabular}{llrrrrrr}
\hline
 scope   & spesific
scope           &       RMSE
[℃] &   MAE [℃] &        bias
[℃] &   log($\kappa$(model)) &    digit
sensitivity &    R² \\
\hline
 global  & ---       & 2.074 &     1.621 &  0.608 &                 -1.271 & -2 & 0.861 \\
 region  & Østfold   & 2.168 &     1.704 &  0.24  &                 -1.256 & -2 & 0.856 \\
 region  & Vestfold  & 2.022 &     1.564 &  0.219 &                 -1.265 & -2 & 0.892 \\
 region  & Trøndelag & 1.957 &     1.528 &  1.235 &                 -1.266 & -2 & 0.782 \\
 region  & Innlandet & 2.165 &     1.71  &  0.714 &                 -1.265 & -2 & 0.873 \\
 local   & 52        & 2.418 &     1.837 & -0.636 &                 -1.257 & -2 & 0.664 \\
 local   & 41        & 1.975 &     1.587 & -0.293 &                 -1.273 & -2 & 0.9   \\
 local   & 37        & 2.206 &     1.755 &  0.373 &                 -1.265 & -2 & 0.873 \\
 local   & 118       & 2.165 &     1.697 &  1.137 &                 -1.266 & -2 & 0.839 \\
 local   & 50        & 1.395 &     1.105 & -0.046 &                 -1.261 & -2 & 0.938 \\
 local   & 42        & 2.239 &     1.75  &  0.333 &                 -1.263 & -2 & 0.881 \\
 local   & 38        & 2.42  &     1.908 &  0.667 &                 -1.264 & -2 & 0.845 \\
 local   & 30        & 1.914 &     1.519 & -0.046 &                 -1.264 & -2 & 0.908 \\
 local   & 57        & 1.978 &     1.547 &  1.108 &                 -1.266 & -2 & 0.85  \\
 local   & 39        & 1.896 &     1.455 &  1.193 &                 -1.269 & -2 & 0.828 \\
 local   & 34        & 2.143 &     1.687 &  1.535 &                 -1.271 & -2 & 0.397 \\
 local   & 15        & 1.806 &     1.428 &  1.114 &                 -1.267 & -2 & 0.787 \\
 local   & 27        & 2.063 &     1.627 &  0.396 &                 -1.256 & -2 & 0.895 \\
 local   & 26        & 2.43  &     1.937 &  1.251 &                 -1.269 & -2 & 0.855 \\
 local   & 17        & 2.26  &     1.78  &  0.921 &                 -1.264 & -2 & 0.875 \\
 local   & 11        & 1.879 &     1.504 &  0.339 &                 -1.261 & -2 & 0.864 \\
\hline
\end{tabular}
	}
	\caption{Daily Plauborg model results.}
	\label{tab:plauborg_day_res}
\end{table}

With modification to the model to accept hourly data it still preforms approximately as well as the daily data version. With a average error of $0.597^\circ C \pm 2.529^\circ C$ for TJM10 and $0.528^\circ C \pm 2.676^\circ C$ for TJM20. It was found that the modified Plauborg model only needs 2 sine terms to make a good prediction and 12h of air temperature which would translate to half a day instead of 3 days.

\subsection{Deep learning models}

LSTM based models has a higher \acrshort{ac:rmse} than the \acrshort{ac:gru} model that performs similary to the Plauborg models in its performance while the rest (BiLSTM, and LSTM) inhibits a Autumm descrepensy, see section \ref{sec:autumn}. 
\begin{figure}[H]
%	\begin{subfigure}{\textwidth}
		\centering
		\includegraphics[width=0.8\linewidth,height=0.3\textheight]{../../results/plots/diffplot_l2KerasBiLSTM_stat_20_Innlandet_2022_TJM20}
		\caption[LSTM TJM10]{LSTM model applied at stations in the region Innlandet in 2022 with 10cm soil temperature as target.}
		\label{fig:diffplotl2kerasbilstmstats10innlandet2022tjm10}
\end{figure}
\begin{figure}
		\centering
		\includegraphics[width=0.8\linewidth,height=0.3\textheight]{../../results/plots/diffplot_KerasGRU_stats_20_Innlandet_2022_TJM20}
		\caption[GRU TJM10]{GRU model applied at staions in the region Innladet in 2022 with 10cm soil temperature as target.}
		\label{fig:diffplotgrustats10innlandet2022tjm10}
\end{figure}
\begin{figure}
		\centering
		\includegraphics[width=0.8\linewidth,height=0.3\textheight]{../../results/plots/diffplot_l1KerasBiLSTM_stat_20_Innlandet_2022_TJM20}
		\caption[BiLSTM TJM10]{BiLSTM applied at stations in the region Innlandet in 2022 with 10cm soil temperature as target.}
		\label{fig:diffplotl1kerasbilstmstats10innlandet2022tjm10}
\end{figure}

The number of epochs was fixed at 10, however the performance grafs shows that after 
\begin{figure}
	\begin{subfigure}{0.45\textwidth}
		\centering
		\includegraphics[width=\linewidth]{../../results/plots/epoch_graf_l1KerasBiLSTM_stat_10}
		\caption[Epoch graph BiLSTM 10cm]{Graf of BiLSTM 10cm performance per epoch.}
		\label{fig:epochgrafl1kerasbilstmstat10}
	\end{subfigure}
	\begin{subfigure}{0.45\textwidth}
		\centering
		\includegraphics[width=\linewidth]{../../results/plots/epoch_graf_l1KerasBiLSTM_stat_20}
		\caption[Epoch graph BiLSTM 20cm]{Graf of BiLSTM 20cm performance per epoch.}
		\label{fig:epochgrafl1kerasbilstmstat20}
	\end{subfigure}
	\begin{subfigure}{0.45\textwidth}
		\centering
		\includegraphics[width=\linewidth]{../../results/plots/epoch_graf_l2KerasBiLSTM_stat_10}
		\caption[Epoch graph LSTM 10cm]{Graf of LSTM 10cm performance per epoch.}
		\label{fig:epochgrafl2kerasbilstmstat10}
	\end{subfigure}
	\begin{subfigure}{0.45\textwidth}
		\centering
		\includegraphics[width=\linewidth]{../../results/plots/epoch_graf_l2KerasBiLSTM_stat_20}
		\caption[Epoch graph LSTM 20cm]{Graf of LSTM 20cm performance per epoch.}
		\label{fig:epochgrafl2kerasbilstmstat20}
	\end{subfigure}
	\caption{Performance graphs displaying the developments of Mean Square Error and \acrfull{ac:r2} for each epoch.}
\end{figure}
