\section{Method}

\subsection{Source of data}

For this comparative study the following data sourses will be used
\begin{enumerate}
	\item Norwegian Institute of Bioeconomy Research LandbruksMeteorologisk service (LMT)
	\item Xgeo
	\item Norwegian Institute of Bioeconomy Research Kilden (Kilden)
	\item The Norwegian Meteorological Institute (MET)
\end{enumerate}

\subsection{Dataset}

The dataset is chosen from four regions in Norway; Innlandet, Vestfold, Trøndelag, and Østfold. From each region are four stations picked:
\begin{itemize}
	\item[Innlandet] \begin{enumerate}
		\item Kise
		\item Ilseng
		\item Apelsvoll
		\item Gausdal
	\end{enumerate}
	\item[Østfold] \begin{enumerate}
		\item Rygge
		\item Rakkestad
		\item Tomo
		\item Øsaker
	\end{enumerate}
	\item[Trøndelag] \begin{enumerate}
		\item Kvithamar
		\item Rissa
		\item Frosta
		\item Mære
	\end{enumerate}
	\item[Vestfold] \begin{enumerate}
		\item Lier
		\item ---
		\item ---
		\item ---
	\end{enumerate}
\end{itemize}

All stations are sampled from the date\footnote{Format month-day} 03-01 to 10-31. The features rain (RR), mean soil temperature at 10cm (TJM10), mean soil temperature at 20cm (TJM20), and air temperature at 2m (TM) are sampled from the LMT database. The snow parameter is sampled from MET via Xgeo for imputed values in areas where there are no messured values. The soil type, and soil texture is sampled from Kilden from Norwegian Institute of Bioeconomy Research.

\subsection{Selection prosess}
The selection prosess for finding these station can be cmopiled into these steps
\begin{enumerate}
	\item recomentation from Norwegian Institute of Bioeconomy Research
	\item \label{list:na_anal}Missing values analyse 
	\item Searshing LMT database for alternative station candidates
	\item Repeat step \ref{list:na_anal}
\end{enumerate}