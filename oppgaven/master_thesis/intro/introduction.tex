\section{Introduction}

In agriculture soil temperature is one of the important parameters to put into consideration when thinking about pest prevention, conservation, and yield prediction. The reasoning for this is that knowing the soil temperature is knowing climate change \cite{li_attention-aware_2022}, water management \cite{alizamir_advanced_2020}, yield \cite{sim_prediction_2020}, nitrogen processes \cite{rankinen_simple_2004} in the soil, calculation of plant-growth \cite{li_modeling_2020}, when seeds start to sprout \cite{li_modeling_2020}, and predicting when insect eggs hatch that were laid last winter. Being able to predict the soil temperature into the future will be a huge advantage for farmers, and scientists.

If it's important, why don't institutions measure it everywhere? There are several reasons for this, but a common reason is that it's expensive to install new equipment on old weather stations. Sometimes the weather station do have the sensors in the fields reading soil temperature at given levels, but due to technical misadventures and unforeseen phenomenons there might be gaps or misreadings that need to be replaced with approximations or NULL values\footnote{These values are different from 0 as they represent "no data" and can't be used to do calculations.}. There are algorithms, models, and statistical tools to interpolate these missing values but they have their drawbacks. For instance approximation by global mean, which is a common method used in timeseries\cite{lepot_interpolation_2017}. This method is perserved global statitics, however does not represent local changes. Further more for a good estimation of soil temperature it is useful to include exogenous\footnote{Variable that can affect the model, but is not not directly described by the model.} features. This is done by the model ARMAX that forcasts a timeseries based on previus seasons and trends but also incoporates external factors if available to improve the model. 

In this study 4 methods will be compared and evaluated for the sake of further research into interpolation of missing data in northic countries based on as few features as possible. This study has chosen 2 types of models; Analytical, and Data-Driven models. 