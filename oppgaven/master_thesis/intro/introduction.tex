\addcontentsline{toc}{section}{Abstracts}
\begin{center}
\begin{minipage}[t]{0.45\textwidth}
	\centering
	\textbf{Abstract}\\
	This study focuses on 3 models that have been used in the literature to predict soil temperatures. The depths chosen as targets are 10cm, and 20cm in 4 regions; Innlandet, Østfold, Vestfold, and Trøndelag. In each region there are 4 stations with self-draining or saturated soils.
\end{minipage}
\begin{minipage}[t]{0.45\textwidth}
	\centering
	\textbf{Oppsummering}\\
	Denne studien ser på modeler som predikerer jordtemperaturer
\end{minipage}
\end{center}	\todo{Fill out when soon done}

% hva gjør vi

% metode

% resultat

\keywords{Soil temperature, Machine learning, regression,hour, weather forecasting data}

\section{Introduction}

In agriculture soil temperature is one of the important parameters to put into consideration when thinking about pest prevention, conservation, and yield prediction. The reasoning for this is that knowing the soil temperature is knowing climate change \cite{li_attention-aware_2022}, water management \cite{alizamir_advanced_2020}, yield \cite{sim_prediction_2020}, nitrogen processes \cite{rankinen_simple_2004} in the soil, calculation of plant-growth \cite{li_modeling_2020}, when seeds start to sprout \cite{li_modeling_2020}, potential flooding and erosions\cite{stuurop_influence_2022}, and predicting when insect eggs hatch that were laid last winter. Being able to predict the soil temperature into the future will be a huge advantage for farmers, civilians, and scientists.

If it's important, why don't institutions measure it everywhere? There are several reasons for this, but a common reason is that it's expensive to install new equipment on old weather stations. Sometimes the weather station do have the sensors in the fields reading soil temperature at given levels, but due to technical misadventures and unforeseen phenomenons there might be gaps or misreadings that need to be replaced with approximations or NULL values\footnote{These values are different from 0 as they represent "no data" and can't be used to do calculations.}. There are algorithms, models, and statistical tools to interpolate these missing values but they have their drawbacks. For instance approximation by global mean, which is a common method used in timeseries\cite{lepot_interpolation_2017}. This method is preserved global statistics, however does not represent local changes. Further more for a good estimation of soil temperature it is useful to include exogenous\footnote{Variable that can affect the model, but is not not directly described by the model.} features.

There has been done research into heat conductivity in soil that has lead to differential equations\cite{karvonen_model_1988}, however these equations\cite{fourier_analytical_2009,karvonen_model_1988} are computationally expensive and difficult to simulate, or calculate\cite{rankinen_simple_2004}. \comment{To add to the complexity the heat dynamics change depending on soil temperature as they change the physical stucture of the soil}{Find source!, if not remove}

There are also massive developments in the types of models, one of which are ASPER that combines logical statments\footnote{Statements can be tought of as formulas, nature laws, knowledge about the solution} with deep learning models to achieve better or similar results to "non-logical" deep learning models, but on fewer samples\cite{le_asper_2023}. A study has been preformed with the model which shows that the model can reduse the number of samples/observations needed by a factor of 1/1000\cite{machot_bridging_2023}. After an intervju with the study resercher (\citeauthor{machot_bridging_2023}), although the model needs a strict ruleset it is possible to incorporate baysioan statistics to make the model more general for more applications by weakening the ruleset and imply that the rules given might not be 100\% accurate and can be relaxed. Another cutting-edge method used is attention-awareness, same method used in ChatGPT, and other modern AI technologies for generative media. This method has been utilised to predict soil temperatures and soil moisture\cite{li_attention-aware_2022}.

A beneficial model would be one using the fewest number of parameters as possible while returning results within acceptable tolerances. This study will consider models that can use only time and air temperature as those two features are the most common measurements measured at weather stations, since soil temperature is not necessarily calculated as stated earlier. A good metric in this study will be considered to be a combination of Root Mean Square Error and Explained Variance (see section \ref{sec:method:metric}). 

The goal of this study is to find which type of model is worth to do further research on. The models selected is a small poll of models in the literature that are being used to predict soil temperature using metrological observations. The scope of metrological is limited to time (as either day of the year, or the hour of the year) as these are common observations in the literature and in Norwegian weather stations. 

